\usepackage{amsmath}
\usepackage{amsfonts}
\usepackage{amssymb}
\usepackage{graphicx}
\usepackage{blindtext}
\usepackage{textcomp}
\usepackage{pgfplots}

\pgfplotsset{width=10cm,compat=1.9}


%Header styles
\usepackage{fancyhdr}
\setlength{\headheight}{15pt}
\pagestyle{fancy}
\renewcommand{\sectionmark}[1]{\markright{#1}{}}
\fancyhf{}
\fancypagestyle{plain}{ %
\fancyhf{} % remove everything
\renewcommand{\headrulewidth}{0pt} % remove lines as well
\renewcommand{\footrulewidth}{0pt}}

%makes available the commands \proof, \qedsymbol and \theoremstyle
\usepackage{amsthm}

%Ruler
\newcommand{\HRule}{\rule{\linewidth}{0.5mm}}

%Commands for naturals, integers, topology, hull, Ball, Disc, Dimension, boundary and a few more
\newcommand{\E}{{\mathcal{E}}}
\newcommand{\F}{{\mathcal{F}}}
\newcommand{\T}{{\mathcal{T}}}
\newcommand{\Bs}{{\mathcal{B}}}
\newcommand{\R}{{\mathbb{R}}}
\newcommand{\Q}{{\mathbb{Q}}}
\newcommand{\Z}{{\mathbb{Z}}}
\newcommand{\Nt}{{\mathbb{N}}}
\newcommand{\B}{{\mathbf{B}}}
\renewcommand{\S}{{\mathbf{S}}}
\newcommand{\K}{{\mathfrak{C}}}
\newcommand{\N}{{\mathfrak{N}}}
\newcommand{\I}{{\mathbf{I}}}
\newcommand{\dime}{{\rm dim}\,}
\newcommand{\est}{{\rm Est}\,}
\newcommand{\inte}{{\rm int}\,}
\newcommand{\conv}{{\rm conv}\,}
\renewcommand{\max}{{\rm sup\,}}
\newcommand{\diam}{{\rm di\acute{a}m\,}}
\newcommand{\leyenda}[1]{\caption{{\small \textsf{#1}}}}
\renewcommand{\inf}{{\rm \acute{i}nf}\,}

\usepackage[margin=1in]{geometry} 
\usepackage{amsmath}
\usepackage{tcolorbox}
\usepackage{indentfirst}
\usepackage{amssymb}
\usepackage{amsthm}
\usepackage{lastpage}
\usepackage{fancyhdr}
\usepackage{enumitem}
\usepackage{accents}
\usepackage{blindtext}
\usepackage{booktabs}
\usepackage{enumitem}
\usepackage{mathtools}
\pagestyle{fancy}
\setlength{\headheight}{40pt}
\usepackage[utf8]{inputenc}
\usepackage[russian]{babel}
\everymath{\displaystyle}
\usepackage{cancel}
\geometry{verbose,a4paper,tmargin=2cm,bmargin=2cm,lmargin=1cm,rmargin=1.5cm}

\usepackage{hyperref}
\hypersetup{
    colorlinks,
    citecolor=black,
    filecolor=black,
    linkcolor=blue,
    urlcolor=blue
}

\newenvironment{solution}
  {\renewcommand\qedsymbol{$\blacksquare$}
  \begin{proof}[Solution]}
  {\end{proof}}
\renewcommand\qedsymbol{$\blacksquare$}

\newtheorem{theorem}{Теорема}
\newtheorem{lemma}{Лемма}
\newtheorem{statement}{Утверждение}
\newtheorem{corollary}{Следствие}
\newtheorem{advice}{Предложение}

\theoremstyle{definition}
\newtheorem{example}{Пример}

\theoremstyle{definition}
\newtheorem{definition}{Определение}

\theoremstyle{remark}
\newtheorem{remark}{Замечание}

\theoremstyle{definition}
\newtheorem*{reminder}{Напоминание}

\newcommand{\ubar}[1]{\underaccent{\bar}{#1}}

\DeclarePairedDelimiter\abs{\lvert}{\rvert}%

\makeatletter
\let\oldabs\abs
\def\abs{\@ifstar{\oldabs}{\oldabs*}}



\begin{document}

\lhead{Иванов Семен} 
\rhead{БПМИ-183} 
\cfoot{\thepage\ of \pageref{LastPage}}


\section{Листок 3. Задача 9}
\begin{itemize}
\item $2\xi_1 \sim N\left(0, 4\right)$, $-3\xi_2 \sim N\left(3, 9\right)$, $\xi_3 \sim N\left(0, 4\right)$, $-\xi_4 \sim N\left(-1, 4\right)$
\item Сумма независимых случайных величин с нормальным распределением $-$ случайная величина с нормальным распределением: $\xi = 2 \xi_1 - 3 \xi_2 + \xi_3 - \xi_4 \sim N(2, 21)$
\item Пусть $\eta \sim N\left(0, 1\right)$, тогда $\eta \sqrt{21} + 2 = \xi$. Получаем:
\[
    P\left(\abs{\xi} < 13\right) = F_{\xi}\left(13\right) - F_{\xi}\left(-13\right) = \Phi\left(\frac{13 - 2}{\sqrt{21}}\right) - \Phi\left(\frac{-13-2}{\sqrt{21}}\right) = \Phi\left(\frac{11}{\sqrt{21}}\right) - \Phi\left(\frac{-15}{\sqrt{21}}\right)
\]
\end{itemize}

\section{Листок 3. Задача 10}
\begin{itemize}
    \item Так как величины независимые: 
\[
    \rho_{\xi, \eta}\left(x, y\right) = \rho_{\xi}\left(x\right) \rho_{\eta}\left(y\right) = \frac{1}{2 \pi} e ^ {-\frac{x ^ 2 + y ^ 2}{2}}
\]
\item Найдем плотность $\frac{\xi}{\eta}$ (так как $F\left(-\infty\right) = 0$ и $F\left(+\infty\right) = 1$, то есть на краях они константы, то мы спокойно можем брать производные от вторых интегралов):
\[
    \rho_{\frac{\xi}{\eta}}(t) = \left(F_\frac{\xi}{\eta}\left(t\right)\right)'_{t} = \left(\int_{0}^{+\infty} \int_{-\infty}^{tx} \rho_{\xi, \eta}\left(x, y\right) dy\ dx\right)'_{t} + \left(\int_{-\infty}^{0} \int_{tx}^{+\infty} \rho_{\xi, \eta}\left(x, y\right) dy\ dx\right)'_{t} = 
\]
\[
    = \int_{0}^{+\infty} x \rho_{\xi, \eta}\left(x, tx\right) dx - \int_{-\infty}^{0} x \rho_{\xi, \eta}\left(x, tx\right)dx = 
    -\frac{1}{2 \pi\left(t ^ 2 + 1\right)}\int^{+\infty}_{0} e ^ {-\frac{\left(t ^ 2 + 1\right)x ^ 2}{2}}d\left(-\frac{\left(t ^ 2 + 1\right) x ^ 2}{2}\right) + 
\]
\[
    + \frac{1}{2 \pi \left(t ^ 2 + 1\right)} \int_{-\infty}^{0} e ^ {-\frac{\left(t ^ 2 + 1\right) x ^ 2}{2}} d\left(-\frac{\left(t ^ 2 + 1\right) x ^ 2}{2}\right) = \frac{1}{\pi \left(t ^ 2 + 1\right)}
\]

\end{itemize}

\section{Листок 3. Задача 4с}
\begin{itemize}
\item Пусть $X = \frac{\xi + \zeta \eta}{\sqrt{1 + \zeta ^ 2}}$
\item Вычислим $\varphi_{X}\left(t\right)$:
\[
    \varphi_{X}\left(t\right) = \mathbb{E} e ^ {itX} = 
\]
Распишем по определению(через интеграл) для $\zeta$ (так как все величины независимые, то их совместная плотность это произведение всех плотностей, то есть все честно):
\[
    = \int_{-\infty}^{+\infty} \frac1{\sqrt{2\pi}} e ^ {-\frac {x ^ 2}{2}} \mathbb{E} e ^ {it \left(\frac{1}{\sqrt{1 + x ^ 2}}\xi + \frac{x}{\sqrt{1 + x ^ 2}} \eta\right)} dx
\]
\item Контстанты $\frac{1}{\sqrt{1 + x ^ 2}}$ и $\frac{x}{\sqrt{1 + x ^ 2}}$ можно воспринимать как синус и косинус, тогда воспользовавшись номером 4а, получаем
\[
    \mathbb{E} e ^ {it \left(\frac{1}{\sqrt{1 + x ^ 2}}\xi + \frac{x}{\sqrt{1 + x ^ 2}} \eta\right)} = e ^ {-\frac{t ^ 2}2}
\]
\item В итоге получаем
\[
    \varphi_{X}\left(t\right) = \int_{-\infty}^{+\infty} \frac1{\sqrt{2\pi}} e ^ {-\frac {x ^ 2}{2}} e ^ {-\frac{t ^ 2}{2}}dx = 
    e ^ {-\frac{t ^ 2}{2}} \int_{-\infty}^{+\infty} \frac1{\sqrt{2\pi}} e ^ {-\frac {x ^ 2}{2}} dx = e ^ {-\frac{t ^ 2}{2}} \cdot 1 = e ^ {-\frac{t ^ 2}{2}}
\]
\item Получили харфункцию $N\left(0, 1\right)$. $\Rightarrow X \sim N\left(0, 1\right)$

\end{itemize}
\qed

\section{Листок 3. Задача 11b}

\begin{itemize}
\item $\xi + \eta \sim -\xi - \eta \sim \xi - \eta \sim -\xi + \eta \sim N\left(0, 8\right)$
\item Раскроем модули по формуле полной вероятности:
\[
    P\left(2 \leq \abs{\xi} + \abs{\eta} \leq 3\right) = P ^ 2 \left(\xi < 0\right) P\left(2 \leq -\xi - \eta \leq 3\right) + 
    \]
    \[ + P ^ 2\left(\xi \geq 0\right) P\left(2 \leq \xi + \eta \leq 3\right) + 2 P\left(\xi < 0\right) P\left(\xi \geq 0\right) P\left(2 \leq \xi - \eta \leq 3\right)
\]
\item Пользуясь функцией распределения стандартного распределения, раскроем все вероятности ($P\left(\xi < 0\right) = P\left(\xi \geq 0\right) = \frac12$, так как $F_{\xi}\left(t\right) = 
\Phi\left(\frac t 2\right)$):
\[
    P\left(2 \leq \abs{\xi} + \abs{\eta} \leq 3\right) = \frac 1 4 \left(\Phi\left(\frac{3 - 0}{2\sqrt{2}}\right) - \Phi\left(\frac{2 - 0}{2\sqrt{2}}\right)\right) + \frac 1 4 \left(\Phi\left(\frac{3 - 0}{2\sqrt{2}}\right) - \Phi\left(\frac{2 - 0}{2\sqrt{2}}\right)\right) + \]\[+ \frac 1 2 \left(\Phi\left(\frac{3 - 0}{2\sqrt{2}}\right) - \Phi\left(\frac{2 - 0}{2\sqrt{2}}\right)\right)= 
 \Phi\left(\frac{3}{2\sqrt{2}}\right) - \Phi\left(\frac{2}{2\sqrt{2}}\right)
\]
\end{itemize}
\end{document}
