\usepackage{amsmath}
\usepackage{amsfonts}
\usepackage{amssymb}
\usepackage{graphicx}
\usepackage{blindtext}
\usepackage{textcomp}
\usepackage{pgfplots}

\pgfplotsset{width=10cm,compat=1.9}


%Header styles
\usepackage{fancyhdr}
\setlength{\headheight}{15pt}
\pagestyle{fancy}
\renewcommand{\sectionmark}[1]{\markright{#1}{}}
\fancyhf{}
\fancypagestyle{plain}{ %
\fancyhf{} % remove everything
\renewcommand{\headrulewidth}{0pt} % remove lines as well
\renewcommand{\footrulewidth}{0pt}}

%makes available the commands \proof, \qedsymbol and \theoremstyle
\usepackage{amsthm}

%Ruler
\newcommand{\HRule}{\rule{\linewidth}{0.5mm}}

%Commands for naturals, integers, topology, hull, Ball, Disc, Dimension, boundary and a few more
\newcommand{\E}{{\mathcal{E}}}
\newcommand{\F}{{\mathcal{F}}}
\newcommand{\T}{{\mathcal{T}}}
\newcommand{\Bs}{{\mathcal{B}}}
\newcommand{\R}{{\mathbb{R}}}
\newcommand{\Q}{{\mathbb{Q}}}
\newcommand{\Z}{{\mathbb{Z}}}
\newcommand{\Nt}{{\mathbb{N}}}
\newcommand{\B}{{\mathbf{B}}}
\renewcommand{\S}{{\mathbf{S}}}
\newcommand{\K}{{\mathfrak{C}}}
\newcommand{\N}{{\mathfrak{N}}}
\newcommand{\I}{{\mathbf{I}}}
\newcommand{\dime}{{\rm dim}\,}
\newcommand{\est}{{\rm Est}\,}
\newcommand{\inte}{{\rm int}\,}
\newcommand{\conv}{{\rm conv}\,}
\renewcommand{\max}{{\rm sup\,}}
\newcommand{\diam}{{\rm di\acute{a}m\,}}
\newcommand{\leyenda}[1]{\caption{{\small \textsf{#1}}}}
\renewcommand{\inf}{{\rm \acute{i}nf}\,}

\usepackage[margin=1in]{geometry} 
\usepackage{amsmath}
\usepackage{tcolorbox}
\usepackage{indentfirst}
\usepackage{amssymb}
\usepackage{amsthm}
\usepackage{lastpage}
\usepackage{fancyhdr}
\usepackage{enumitem}
\usepackage{accents}
\usepackage{blindtext}
\usepackage{booktabs}
\usepackage{enumitem}
\usepackage{mathtools}
\pagestyle{fancy}
\setlength{\headheight}{40pt}
\usepackage[utf8]{inputenc}
\usepackage[russian]{babel}
\everymath{\displaystyle}
\usepackage{cancel}
\geometry{verbose,a4paper,tmargin=2cm,bmargin=2cm,lmargin=1cm,rmargin=1.5cm}

\usepackage{hyperref}
\hypersetup{
    colorlinks,
    citecolor=black,
    filecolor=black,
    linkcolor=blue,
    urlcolor=blue
}

\newenvironment{solution}
  {\renewcommand\qedsymbol{$\blacksquare$}
  \begin{proof}[Solution]}
  {\end{proof}}
\renewcommand\qedsymbol{$\blacksquare$}

\newtheorem{theorem}{Теорема}
\newtheorem{lemma}{Лемма}
\newtheorem{statement}{Утверждение}
\newtheorem{corollary}{Следствие}
\newtheorem{advice}{Предложение}

\theoremstyle{definition}
\newtheorem{example}{Пример}

\theoremstyle{definition}
\newtheorem{definition}{Определение}

\theoremstyle{remark}
\newtheorem{remark}{Замечание}

\theoremstyle{definition}
\newtheorem*{reminder}{Напоминание}

\newcommand{\ubar}[1]{\underaccent{\bar}{#1}}

\DeclarePairedDelimiter\abs{\lvert}{\rvert}%

\makeatletter
\let\oldabs\abs
\def\abs{\@ifstar{\oldabs}{\oldabs*}}



\begin{document}
\lhead{Иванов Семен} 
\rhead{БПМИ-183} 
\cfoot{\thepage\ of \pageref{LastPage}}

\section*{Листок 1. Задача 9.}
\begin{itemize}
    \item Пусть $\xi_{i} \ -$ равномерное распределенная случайная величина, отвечающая за время приема $i$-того пациента. Тогда $E \xi_{i} = \frac{1 + 4}{2} = 2.5$, $D \xi_{i} = \frac{\left(4 - 1 \right) ^ 2}{12} = \frac{9}{12}$.
    \item Теперь введем случайную величину, отвечающую за суммарное время приема:\\ $\eta = \xi_{1} + \xi_{2} + \xi_{3} + \xi_{4}$. $E\eta = 4 \cdot 2.5 = 10$. Так как $\xi_{i}$ не зависят друг от друга, можем просто найти дисперсию $D \eta = 4 \cdot \frac{9}{12} = 3$. 
    \item Теперь применим неравенство Чебышева к $\eta$: \\ $P\left(\left| \eta - 10\right| \geq 2\right) = P\left(\eta \geq 12\right) \, + \, P\left(\eta \leq 8\right) \leq \frac{D \eta}{2 ^ 2} = \frac{3}{4}$
    \item Допустим, у нас есть две плотности $\rho_{1}$ и $\rho_{2}$, которые симметричны относительно некторой точки $m$ (не ограничивая общности, будем считать, что $m = 0$), тогда по формуле свертки посчитаем плотность суммы этих случайных величин: $\rho\left(t\right) = \int_{- \infty}^{+ \infty} \rho_{1}\left(x\right) \rho_{2}\left(t - x\right) dx = \int_{- \infty}^{+ \infty} \rho_{1}\left(-x\right) \rho_{2}\left((-t) - (-x)\right)dx = \rho\left(-t \right)$. То есть плотность суммы этих случайных величин тоже симметрична относительно $m$. Таким образом плотность $\xi_{1} + \xi_{2} + \xi_{3} + \xi_{4}$ симметрична относительно 10. То есть $P\left(\eta \leq 8\right) = P\left(\eta \geq 12\right)$
    \item Получаем $P\left(\eta \geq 12 \right) \leq \frac{1}{2} \cdot P\left(\left| \eta - 10\right| \geq 2\right) \leq \frac{3}{8}$.
\end{itemize}
\section*{Листок 1. Задача 10.}
\begin{itemize}
    \item Найдем распределение $F_{m_{n}}(t) = P\left(m_{n} < t\right) = 1 - P\left(m_{n} > t\right) = \begin{cases} 0 & t < 0 \\
    1 - \left(1 - t\right) ^ n & 0 \leq t \leq 1 \\
    1 & 1 < t
    \end{cases}$
    \item $m_{n} \geq 0$, $m_{n + 1} \leq m_{n}$ $\Rightarrow$ $\exists \lim_{n \to \infty} m_{n} = m$.
    \item Надо доказать, что $P(m = 0) = 1 = F_m(0)$.
    \item Рассмотрим систему вложенных множеств $A_n = \left[0, \frac{1}{n}\right]$, $\ A = \bigcap_{n = 1}^{\infty} A_{n} = \{0\}$. Из непрерывности веротяностной меры знаем, что: \\ $P\left(m \in A\right) = P(m = 0) = \lim_{n \to \infty} P\left(m \in A_{n}\right) = \lim_{n \to \infty} P\left(0 \leq m \leq \frac{1}{n}\right)$
    \item $\lim_{n \to \infty} P\left(0 \leq m \leq \frac{1}{n}\right) = \lim_{n \to \infty} F_{m}\left(\frac{1}{n}\right) =\lim_{n \to \infty} \lim_{k \to \infty} F_{m_{k}}\left(\frac{1}{n}\right) = \lim_{n \to \infty} \lim_{k \to \infty} \left(1 - \left(1 - \frac{1}{n}\right) ^ k\right) = $\\$=\lim_{n \to \infty} 1 = 1 = P\left(m = 0\right)$. \qed{}
\end{itemize}
\section*{Листок 1. Задача 11.}
\begin{itemize}
    \item Рассмотрим сходимость $\ln \eta_{n} = \ln \sqrt[n]{\xi_{1} \cdots \xi_{n}} = \frac{1}{n} \cdot \left(\ln \xi_{1} + \cdots + \ln \xi_{n}\right)$.
    \item $\ln \xi_1, \ln \xi_{2}, \cdots \ -$ последовательность одинаково распределенных, попарно независимых случайных величин с конечным вторым моментом:
    \\$E \ln ^ 4 \xi_{1} = \int_{0}^{1} \ln ^ 4 x dx < \infty$ (так как $\int_{0}^{1} \ln x dx $ сходится, и там $\ln x < 0$ везде)
    \item Применяя закон больших чисел, получаем, что \\$ \lim_{n \to \infty}\frac{1}{n} \cdot \left(\ln \xi_{1} + \cdot + \ln \xi_{n}\right) = E \ln \xi_{1} = \int_{0}^{1} 1 \cdot \ln x dx = -1$
    \item То есть $\lim_{n \to \infty} \ln \eta_n = -1$. Следовательно $\lim_{n \to \infty} \eta_{n} = \frac{1}{e}$
\end{itemize}
\end{document}