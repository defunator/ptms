\usepackage{amsmath}
\usepackage{amsfonts}
\usepackage{amssymb}
\usepackage{graphicx}
\usepackage{blindtext}
\usepackage{textcomp}
\usepackage{pgfplots}

\pgfplotsset{width=10cm,compat=1.9}


%Header styles
\usepackage{fancyhdr}
\setlength{\headheight}{15pt}
\pagestyle{fancy}
\renewcommand{\sectionmark}[1]{\markright{#1}{}}
\fancyhf{}
\fancypagestyle{plain}{ %
\fancyhf{} % remove everything
\renewcommand{\headrulewidth}{0pt} % remove lines as well
\renewcommand{\footrulewidth}{0pt}}

%makes available the commands \proof, \qedsymbol and \theoremstyle
\usepackage{amsthm}

%Ruler
\newcommand{\HRule}{\rule{\linewidth}{0.5mm}}

%Commands for naturals, integers, topology, hull, Ball, Disc, Dimension, boundary and a few more
\newcommand{\E}{{\mathcal{E}}}
\newcommand{\F}{{\mathcal{F}}}
\newcommand{\T}{{\mathcal{T}}}
\newcommand{\Bs}{{\mathcal{B}}}
\newcommand{\R}{{\mathbb{R}}}
\newcommand{\Q}{{\mathbb{Q}}}
\newcommand{\Z}{{\mathbb{Z}}}
\newcommand{\Nt}{{\mathbb{N}}}
\newcommand{\B}{{\mathbf{B}}}
\renewcommand{\S}{{\mathbf{S}}}
\newcommand{\K}{{\mathfrak{C}}}
\newcommand{\N}{{\mathfrak{N}}}
\newcommand{\I}{{\mathbf{I}}}
\newcommand{\dime}{{\rm dim}\,}
\newcommand{\est}{{\rm Est}\,}
\newcommand{\inte}{{\rm int}\,}
\newcommand{\conv}{{\rm conv}\,}
\renewcommand{\max}{{\rm sup\,}}
\newcommand{\diam}{{\rm di\acute{a}m\,}}
\newcommand{\leyenda}[1]{\caption{{\small \textsf{#1}}}}
\renewcommand{\inf}{{\rm \acute{i}nf}\,}

\usepackage[margin=1in]{geometry} 
\usepackage{amsmath}
\usepackage{tcolorbox}
\usepackage{indentfirst}
\usepackage{amssymb}
\usepackage{amsthm}
\usepackage{lastpage}
\usepackage{fancyhdr}
\usepackage{enumitem}
\usepackage{accents}
\usepackage{blindtext}
\usepackage{booktabs}
\usepackage{enumitem}
\usepackage{mathtools}
\pagestyle{fancy}
\setlength{\headheight}{40pt}
\usepackage[utf8]{inputenc}
\usepackage[russian]{babel}
\everymath{\displaystyle}
\usepackage{cancel}
\geometry{verbose,a4paper,tmargin=2cm,bmargin=2cm,lmargin=1cm,rmargin=1.5cm}

\usepackage{hyperref}
\hypersetup{
    colorlinks,
    citecolor=black,
    filecolor=black,
    linkcolor=blue,
    urlcolor=blue
}

\newenvironment{solution}
  {\renewcommand\qedsymbol{$\blacksquare$}
  \begin{proof}[Solution]}
  {\end{proof}}
\renewcommand\qedsymbol{$\blacksquare$}

\newtheorem{theorem}{Теорема}
\newtheorem{lemma}{Лемма}
\newtheorem{statement}{Утверждение}
\newtheorem{corollary}{Следствие}
\newtheorem{advice}{Предложение}

\theoremstyle{definition}
\newtheorem{example}{Пример}

\theoremstyle{definition}
\newtheorem{definition}{Определение}

\theoremstyle{remark}
\newtheorem{remark}{Замечание}

\theoremstyle{definition}
\newtheorem*{reminder}{Напоминание}

\newcommand{\ubar}[1]{\underaccent{\bar}{#1}}

\DeclarePairedDelimiter\abs{\lvert}{\rvert}%

\makeatletter
\let\oldabs\abs
\def\abs{\@ifstar{\oldabs}{\oldabs*}}




\title{Теория вероятности и математическая статистика, 2 курс, 2 семестр}
\author{@defunator}
\begin{document}

\rhead{БПМИ-183} 
\cfoot{\thepage\ of \pageref{LastPage}}
\maketitle


\tableofcontents

\clearpage

\section{Сходимости случайных величин}

\begin{definition}
Последовательность случайных величин $\xi_{n}$ сходится к случайной величине $\xi$:
\begin{enumerate}
\item \textit{Почти наверное}($\xi_{n} \xrightarrow{\text{п.н.}} \xi$), если 
\[
    P\left(\lim_{n \to \infty} \xi_{n} = \xi\right) = 1
\]
\item \textit{По вероятности}($\xi_{n} \xrightarrow{\text{p}} \xi$), если
\[
    \forall \varepsilon > 0 \ \lim_{n \to \infty} P\left(|\xi_{n} - \xi| \geq \varepsilon\right) = 0
\]
\item \textit{По распределению}($\xi_{n} \xrightarrow{\text{d}} \xi$), если 
\[
    \lim_{n \to \infty} F_{\xi_{n}}\left(x\right) = F_{\xi}\left(x\right)
\]
для любых $x$, в которых непрерывна $F_{\xi}$
\end{enumerate}
\end{definition}

\begin{theorem}(Эквивалетное определение сходимости по распределению)
$\xi_{n} \xrightarrow{d} \xi$  $\Leftrightarrow$ $\forall g \ -$ непрерывна и ограничена на $\mathbb{R}$ верно
$\lim_{n \to \infty}  \mathbb{E}  g\left(\xi_{n}\right) =  \mathbb{E}  g\left(\xi\right)$

\end{theorem}
\begin{proof}
$\Rightarrow$ \\

Пусть $t \ -$ точка непрерывности $F_{\xi}\left(t\right)$. Заметим, что $F_{\xi}\left(t\right) = P\left(\xi_{x} \in (-\infty, t]\right)=$\\$ = \mathbb{E}  \text{Ind}_{(-\infty, t]}\left(\xi\right)$. \\
В силу:\\
(1) $\mathbb{E} \text{Ind}_{(a_i, b_i]}(\xi_n) = F_{\xi_n}(b_i) - F_{\xi_n}(a_i) \xrightarrow{n \xrightarrow{} \infty} F_{\xi}(b_i) - F_{\xi}(a_i) =\mathbb{E} \text{Ind}_{(a_i, b_i]}(\xi)$\\
(2) Линейность предела (с какими-то коэффициентами $c_i$)\\
Верна следующая сходимость:
\[
    \mathbb{E}\sum\limits_{i = 1}^{N} c_i \cdot \text{Ind}_{(a_i, b_i]}(\xi_n) \xrightarrow{} \mathbb{E}\sum\limits_{i = 1}^{N} c_i \cdot \text{Ind}_{(a_i, b_i]}(\xi)
\]
Теперь нам бы хотелось от непрерывной ограниченной функции на прямой перейти к функции на отрезке, а там мы уже сможем ее приблизить ступенчатой и воспользоваться предыдудщим утверждением и все доказать. Мы знаем, что \\ $\forall \varepsilon > 0 \ \exists A$: $P\left(-A < \xi \leq A\right) > 1 - \varepsilon$ (потому что $P\left(\xi \in \mathbb{R}\right) = 1$). Тогда получаем: 
$$F_{\xi_{n}}\left(A\right) - F_{\xi_{n}}\left(-A\right) \xrightarrow{n \to \infty} F_{\xi}\left(A\right) - F_{\xi}\left(-A\right) = P\left(-A < \xi \leq A\right) > 1 - \varepsilon$$ То есть для $\forall \varepsilon > 0 \ \exists N \ \forall n > N$ верно:
$$\abs{\left(F_{\xi_{n}}\left(A\right) - F_{\xi_{n}}\left(-A\right)\right) - \left(F_{\xi}\left(A\right) - F_{\xi}\left(-A\right)\right)} < \varepsilon$$ 
Комбинируя последние два утверждения, получаем для $\forall \varepsilon > 0\ \exists A \ \exists N \ \forall n > N$: $$F_{\xi_n}\left(A\right) - F_{\xi_n}\left(-A\right) > F_{\xi}\left(A\right) - F_{\xi}\left(-A\right) - \varepsilon > 1 - 2 \varepsilon$$ 
Из чего следует $\forall \varepsilon > 0  \ \exists A \ \exists N\ \forall n > N$: 
\[
    P\left(-A \leq \xi_{n} \leq A\right) \geq F_{\xi_{n}}\left(A\right) - F_{\xi_{n}}\left(-A\right) > 1 - \varepsilon
\]
Теперь возьмем любую непрерывную ограниченную функцию $g$, приблизим ее на отрезке $\left[-A, A\right]$ ступенчатой функцией $g_{\varepsilon}$, что $\abs{g\left(x\right) - g_{\varepsilon}\left(x\right)} < \varepsilon$, а вне отрезка положим $g_{\varepsilon} = 0$. Имеем $\forall \varepsilon > 0 \ \exists A \ \forall n$: 
\[
\abs{ \mathbb{E}  g\left(\xi_{n}\right) -  \mathbb{E}  g\left(\xi\right)} \leq \abs{ \mathbb{E}  \left(1 - \text{Ind}_{\left[-A, A\right]}\left(\xi_{n}\right)\right) \cdot g\left(\xi_{n}\right) -  \mathbb{E}  \left(1 - \text{Ind}_{\left[-A, A\right]}\left(\xi\right)\right)g\left(\xi\right)} + 
\]
\[
    + \abs{ \mathbb{E}  \text{Ind}_{\left[-A, A\right]}\left(\xi_{n}\right) \cdot g\left(\xi_{n}\right) -  \mathbb{E}  \text{Ind}_{\left[-A, A\right]}\left(\xi\right) \cdot g\left(\xi\right)}
\]
Ясно, что первый модуль $< C \cdot 2\varepsilon$ (из ограниченности $g$ $\forall x_1, \ x_2$: $\abs{g\left(x_1\right) - g\left(x_2\right)} < C$ и так как $P\left(\abs{\xi} > A\right), \ P\left(\abs{\xi_{n}} > A\right) < \varepsilon$). А во втором модуле $g$ заменим на $g_{\varepsilon}$ с погрешностью $\varepsilon$, то есть он 
$< 2\varepsilon + \abs{ \mathbb{E}  \text{Ind}_{\left[-A, A\right]}\left(\xi_{n}\right) \cdot g_{\varepsilon}\left(\xi_{n}\right) -  \mathbb{E}  \text{Ind}_{\left[-A, A\right]}\left(\xi\right) \cdot g_{\varepsilon}\left(\xi\right)}$. А про оставшийся модуль мы уже знаем, что он сходится к 0, так как ступенчатая функция(то есть $< \varepsilon$ для $\forall n > N$). В итоге имеем:
$\abs{ \mathbb{E}  g\left(\xi_{n}\right) -  \mathbb{E}  g\left(\xi\right)} < \varepsilon$.
То есть $\lim_{n \to \infty}  \mathbb{E}  g\left(\xi_{n}\right) =  \mathbb{E}  g\left(\xi\right)$ \\
$\Leftarrow$ \\
Пусть $t \ -$ точка непрерывности $F_{\xi}\left(t\right)$. Мы знаем, что $F_{\xi}\left(t\right) =  \mathbb{E}  \text{Ind}_{(-\infty, t]}\left(\xi\right)$.
Для $\forall \delta > 0$ определим функции: 
\[
    g_{- \delta}\left(x\right) = 
    \begin{cases}
        1 & x < t - \delta \\
        \frac{1}{\delta} \cdot \left(t - x\right) & t - \delta \leq x \leq t \\
        0 & t < x
    \end{cases}
\]
\[
    g_{+ \delta}\left(x\right) = 
    \begin{cases}
        1 & x < t \\
        \frac{1}{\delta} \cdot \left(t - x\right) & t \leq x \leq t + \delta \\
        0 & t + \delta < x
    \end{cases}
\]
Заметим, что $\forall x$: $$
\text{Ind}_{(- \infty, t - \delta]}\left(x\right) \leq g_{-\delta}\left(x\right) \leq \text{Ind}_{(-\infty, t]}\left(x\right) \leq g_{+\delta}\left(x\right) \leq \text{Ind}_{(\infty, t + \delta]}\left(x\right)$$
Взяв матожидания ($x = \xi_{n}$) от второго и третьего неравенств, получим: $$ \mathbb{E}  g_{-\delta}\left(\xi_{n}\right) \leq F_{\xi_{n}}\left(t\right) \leq  \mathbb{E}  g_{+ \delta}\left(\xi_{n}\right)$$
Теперь устремим $n \to \infty$: 
\[
     \mathbb{E}  g_{-\delta}\left(\xi\right) \leq \inf \lim_{n \to \infty} F_{\xi_{n}}\left(x\right) \leq \sup \lim_{n \to \infty} F_{\xi_{n}}\left(x\right) \leq  \mathbb{E}  g_{+ \delta}\left(x\right)
\]
Рассмотрим первое и последнее неравенство той цепочки($x = \xi$ и возьмем то него матожидание), получим: 
\[
    F_{\xi}\left(t - \delta\right) \leq  \mathbb{E}  g_{- \delta}\left(\xi\right) \leq  \mathbb{E}  g_{+ \delta}\left(\xi\right) \leq F_{\xi}\left(t + \delta\right)
\]
Теперь $\delta \to 0$: 
\[
    F_{\xi}\left(t\right) =  \mathbb{E}  g_{- \delta}\left(\xi\right) = inf \lim_{n \to \infty} F_{\xi_{n}}\left(t\right) = sup \lim_{n \to \infty} F_{\xi_{n}}\left(t\right) =  \mathbb{E}  g_{+ \delta}\left(\xi\right)
\]
Получаем: $F_{\xi}\left(t\right) = \lim_{n \to \infty} F_{\xi_{n}}\left(t\right)$

\end{proof}

\begin{theorem}
$
    \xi_{n} \xrightarrow{\text{п.н.}} \xi \Rightarrow \xi_{n} \xrightarrow{\text{p}} \xi
$
\end{theorem}
\begin{proof}
Знаем $P\left(\lim_{n \to \infty} \xi_{n} = \xi\right) =1 $. Заметим вложенность следующих событий для $\forall \varepsilon > 0$: $\left\{\lim_{n \to \infty} \xi_{n} = \xi\right\} \Rightarrow \bigcup_{N = 1}^{\infty} \bigcap_{n = N}^{\infty} \left\{\abs{\xi_{n} - \xi} < \varepsilon\right\}$ (это по сути и есть определение предела, что для $\forall \varepsilon$, начиная с некоторого $N$ выполнятется условие). То есть: 
$$1 = P\left(\lim_{n \to \infty} \xi_{n} = \xi\right) \leq P\left(\bigcup_{N = 1}^{\infty} \bigcap_{n = N}^{\infty} \left\{\abs{\xi_{n} - \xi} < \varepsilon\right\}\right) = 1$$ 
Так как последовательность множеств $A_{N} = \bigcap_{n = N}^{\infty} \left\{\abs{\xi_{n} - \xi} < \varepsilon\right\}$ расширяющаяся ($A_{N} \subseteq A_{N + 1}$), в объединении они дают событие вероятности 1, значит по теореме непрерывности $\lim_{N \to \infty} P\left(A_{N}\right) = 1$. 
Теперь заметим: $P\left(A_{N}\right) \leq P\left(\abs{\xi_{N} - \xi} < \varepsilon\right)$.
То есть: $\lim_{N \to \infty} P\left(\abs{\xi_{N} - \xi} < \varepsilon\right) = 1$.
Или что то же самое: $\lim_{N \to \infty} \left(1-P\left(\abs{\xi_{N} - \xi} < \varepsilon\right)\right) = \lim_{N \to \infty} P\left(\abs{\xi_{N} - \xi} \geq \varepsilon\right) = 0 \ -$ определение сходимости по вероятности.
\end{proof}
\begin{theorem}(Теорема Лебега о мажорируемой сходимости) 
$\xi_{n} \xrightarrow{\text{p}} \xi$ и $\abs{\xi_{n}}, \abs{\xi} \leq \eta$ п. н. (где $\eta \ -$ случайная величина, что $ \mathbb{E}  \eta < \infty$), то $ \mathbb{E}  \xi_{n} \to  \mathbb{E}  \xi$
\end{theorem}
\begin{proof}
Докажем теорему в частном случае, когда $\eta \equiv C$.
$\forall \varepsilon > 0 \forall n $ $\abs{ \mathbb{E}  \xi_{n} -  \mathbb{E}  \xi} \leq  \mathbb{E}  \abs{\xi_{n} - \xi} =  \mathbb{E}  \abs{\xi_{n} - \xi} \cdot \text{Ind}_{\abs{\xi_{n} - \xi} \geq \varepsilon} +  \mathbb{E}  \abs{\xi_{n} - \xi} \cdot \text{Ind}_{\abs{\xi_{n} - \xi} < \varepsilon} \leq 2C \cdot P\left(\abs{\xi_{n} - \xi} \geq \varepsilon\right) + \varepsilon \cdot 1$.
Так как $\xi_{n} \xrightarrow{\text{p}} \xi$, то $\exists N \forall n > N$: $P\left(\abs{\xi_n - \xi} \geq \varepsilon\right) < \varepsilon$.
Тогда получаем: $\abs{ \mathbb{E}  \xi_{n} -  \mathbb{E}  \xi} < 2C \cdot \varepsilon + \varepsilon$. То есть $ \mathbb{E}  \xi_n \to  \mathbb{E}  \xi$.

\end{proof}

\begin{advice}
$\xi_{n} \xrightarrow{\text{p}} \xi $ $\Rightarrow$ для $\forall g \ -$ непрерывная, $g\left(\xi_{n}\right) \xrightarrow{\text{p}} g\left(\xi\right)$
\end{advice}
\begin{proof}
Знаем для любой случайной величины $\forall \varepsilon > 0 \exists A$: $P\left(\abs{\xi} > \frac{A}{2}\right) < \varepsilon$.
$\exists N \forall n > N$: $P\left(\abs{\xi_{n}} > A\right) \leq P\left(\abs{\xi - \xi_{n}} + \abs{\xi} > A\right) \leq P\left(\abs{\xi - \xi_{n}} > \frac{A}{2}\right) + P\left(\abs{\xi} > \frac{A}{2}\right) < \varepsilon$.
Теперь возьмем $g$, она равномерно непрерывна на $\left[-A, A\right]$:
\[
    \forall \varepsilon > 0 \exists \delta > 0: \ \abs{x - y} < \delta \Rightarrow \abs{g\left(x\right) - g\left(y\right)} < \varepsilon \ \forall x, y \in \left[-A, A\right]
\]
Докажем $g\left(\xi_{n}\right) \xrightarrow{\text{p}} g\left(\xi\right)$:
\[
    P\left(\abs{g\left(\xi_n\right) - g\left(\xi\right)} \geq \varepsilon\right) = P\left(\abs{g\left(\xi_n\right) - g\left(\xi\right)} \geq \varepsilon \ |\  \xi_n, \xi \in \left[-A, A\right]\right) \cdot P\left(\xi_n, \xi \in \left[-A, A\right]\right) +
    \]
    \[+ P\left(\abs{g\left(\xi_n\right) - g\left(\xi\right)} \geq \varepsilon \ |\  \xi_n, \xi \not \in \left[-A, A\right]\right) \cdot P\left(\xi_n, \xi \not \in \left[-A, A\right]\right)
\]
Посмотрев на определение равномерное непрерывности, заметим, что: 
\[
P\left(\abs{g\left(\xi_n\right) - g\left(\xi\right)} \geq \varepsilon \ |\  \xi_n, \xi \in \left[-A, A\right]\right) \leq P\left(\abs{\xi_n - \xi} \geq \delta\right)
\]
 А это уже, так как у нас есть сходимость по вероятности $\to 0$.
 И заметим, что 
 \[
    P\left(\xi_n, \xi \not \in \left[-A, A\right]\right) \leq P\left(\abs{\xi_n} > A\right) + P\left(\abs{\xi} > A\right) < 2 \varepsilon
 \]
 То есть $\to 0$ при $n, A \to \infty$
Все, получили, что $P\left(\abs{g\left(\xi_n\right) - g\left(\xi\right)} \geq \varepsilon\right) \to 0$ при  \\ $n, A \to \infty$.
\end{proof}

\begin{corollary}
$\xi_{n} \xrightarrow{\text{p}} \xi \Rightarrow \xi_{n} \xrightarrow{\text{d}} \xi$
\end{corollary}
\begin{proof}
Берем предыдущее предложение. Потом используем теорему Лебега для произвольной непрерывной ограниченной $g$ и вспоминаем эквивалентное опеределение сходимости по распределению.
\end{proof}

\begin{theorem} (Эквивалетное опеределение сходимости почти наверное) \\
$$\xi_n \xrightarrow{\text{п.н.}} \xi \Leftrightarrow \forall \varepsilon > 0 : \ \lim_{n \to \infty} P\left(\sup_{k \geq n} \abs{\xi_k - \xi} > \varepsilon\right) = 0$$
\end{theorem}
\begin{proof}
Рассмотрим следующие события:
$$A_{k}^{\varepsilon} = \left\{\abs{\xi_k - \xi} > \varepsilon \right \}$$
$$A^{\varepsilon} = \bigcap_{n = 1}^{\infty} \bigcup_{k \geq n} A_{k}^{\varepsilon}$$
Заметим, что:
$$\left\{\sup_{k \geq n} \abs{\xi_{k} - \xi} > \varepsilon\right\} = \left\{\exists k \geq n: \ \abs{\xi_k - \xi} >\varepsilon\right\} = \bigcup_{k \geq n} A_{k}^{\varepsilon}$$
$$\left\{\lim_{n \to \infty} \xi_{n} \neq \xi\right\} = \left\{\exists \varepsilon > 0 \forall n \exists k \geq n : \ \abs{\xi_{k} - \xi} \geq \varepsilon\right\} = \bigcup_{m = 1}^{\infty} \bigcap_{n}^{\infty} \bigcup_{k \geq n} A_{k}^{\varepsilon = \frac1m} = \bigcup_{m = 1}^{\infty} A^{\frac1m}$$
Тогда имеем:
$$\xi_n \xrightarrow{\text{п.н.}} \xi \Leftrightarrow P\left(\lim_{n \to \infty} \xi_{n} \neq \xi\right) = 0 \Leftrightarrow
P\left(\bigcup_{m = 1}^{\infty} A^{\frac1m}\right) = 0 \Leftrightarrow \forall m \ P\left(A^{\frac1m}\right) = 0 \Leftrightarrow
\forall \varepsilon > 0 \ P\left(A^{\varepsilon}\right) = 0 \Leftrightarrow$$
Теперь заметим вложенность последовательности событий $B_{n}^{\varepsilon} = \bigcup_{k \geq n} A_{k}^{\varepsilon}$ и, взглянув на определение $A^{\varepsilon}$, по теореме о непрерывности вероятностной меры продолжаем цепочку:
$$\Leftrightarrow \lim_{n \to \infty} P\left(\bigcup_{k \geq n} A_{k}^{\varepsilon}\right) = 0 \Leftrightarrow
\lim_{n \to \infty} P\left(\sup_{k \geq n} \abs{\xi_{k} - \xi} > \varepsilon\right) = 0$$
\end{proof}

Теперь приведем некоторые примеры, опровергающие остальные следствия сходимостей
\begin{example}
$$\xi_{n} \xrightarrow{\text{d}} \xi \not\Rightarrow \xi_n \xrightarrow{\text{P}} \xi$$
Пусть $\Omega = \left\{\omega_{1}, \omega_2\right\}$, $P\left(\left\{\omega_i\right\}\right) = \frac12$. Определим $\forall n$ $\xi_n\left(\omega_i\right) = \left(-1\right) ^ i$. Положим $\xi = -\xi_n$. Тогда $\forall f$ непрерыной и ограниченной:
$$ \mathbb{E}  f\left(\xi_n\right) = \frac{f\left(1\right) + f\left(-1\right)}{2} =  \mathbb{E}  f\left(\xi\right)$$
То есть $\xi_{n} \xrightarrow{\text{d}} \xi$. Однако $\forall n \ \abs{\xi_n - \xi} = 2$, то есть $\xi_n \not\xrightarrow{\text{P}} \xi$
\end{example}
\begin{example}
$$\xi_{n} \xrightarrow{\text{P}} \xi \not\Rightarrow \xi_n \xrightarrow{\text{п.н.}} \xi$$
Возьмем $\Omega = \left[0, 1\right]$, $\xi_{2 ^ n + p} = \text{Ind}_{\left[\frac{p}{2 ^ n}, \frac{p + 1}{2 ^ n}\right]}$, $0 \leq p < 2^ n$. Ясно, что $\xi_{n} \xrightarrow{\text{P}} \xi$, так как 
$$\lim_{n \to \infty} P\left(\xi_{n} > 0\right) = \lim_{n \to \infty} \frac{1}{2 ^ {\lfloor \log n \rfloor}} = 0$$, но 
$\xi_n \not\xrightarrow{\text{п.н.}} 0$, так как для любого исхода $\omega$ существует бесконечно много $n$, что $\xi_{n}\left(\omega\right) = 1$. Теперь осталось посмотреть на теорему 4 и все станет ясно.
\end{example}
\begin{example}
$$\xi_{n} \xrightarrow{\text{d}} \xi \not\Rightarrow \xi_n \xrightarrow{\text{п.н.}} \xi$$
Если бы следствие имело место, то отсюда вытекало бы, что:
$$\xi_{n} \xrightarrow{\text{d}} \xi \not\Rightarrow \xi_n \xrightarrow{\text{P}} \xi$$
противоречие

\end{example}
\clearpage





\section{Характеристические функции}

\begin{definition}
\textit{Характеристической функцией} случайной величины $\xi$ называется функция:
$$\varphi_{\xi}\left(t\right) =  \mathbb{E}  e ^ {it \xi} =  \mathbb{E}  \left(\cos\left(t \xi\right) + i \sin\left(t \xi\right)\right)$$
\end{definition}

\begin{advice}
(Свойства характеристических функций)
\begin{enumerate}
\item $\varphi_\xi\left(0\right) = 1$, $\abs{\varphi_\xi\left(t\right)} \leq 1$ $\forall t \in \mathbb{R}$
\item $\varphi_{a\xi+b}\left(t\right) = e ^ {itb} \varphi_\xi\left(at\right)$
\item если $\xi_1, \cdots \xi_n \ - $ независимые случайные величины и $S = \xi_1 + \cdots + \xi_n$, то 
\[
    \varphi_S\left(t\right) = \varphi_{\xi_1}\left(t\right) \cdots \varphi_{\xi_n}\left(t\right)
\]
\end{enumerate}
\end{advice}
\begin{proof}
.
\begin{enumerate}
\item Понятно, так как в матожидании могут быть только комплексные числа с модулем $\leq 1$.
\item $\varphi_{a\xi+b}\left(t\right) =  \mathbb{E}  e ^ {it \left(a\xi + b\right)} = e ^ {itb}  \mathbb{E}  e ^ {i \left(at\right) \xi} = e ^ {itb} \varphi_{\xi}\left(at\right)$
\item $\varphi_{S}\left(t\right) =  \mathbb{E}  \left(e ^ {it \xi_1} \cdots e ^ {it \xi_n}\right) =  \mathbb{E}  e ^ {it\xi_1} \cdots  \mathbb{E}  e ^ {it \xi_n} = \varphi_{\xi_1}\left(t\right) \cdots \varphi_{\xi_n}\left(t\right)$
\end{enumerate}
\end{proof}

\begin{example}
Вычислим $\varphi_{\xi}\left(t\right)$, где $\xi \sim N\left(0, 1\right)$:
\[
    \varphi_{\xi}\left(t\right) =  \mathbb{E}  e ^ {it\xi} = \frac{1}{\sqrt{2 \pi}} \int_{-\infty}^{+\infty} \cos\left(tx\right) e ^ {-\frac{x ^2}{2}} dx + \frac{i}{\sqrt{2 \pi}} \int_{-\infty}^{+\infty} \sin\left(tx\right) e ^ {-\frac{x ^2}{2}} dx
\]
Заметим, что $\int_{-\infty}^{+\infty} \sin\left(tx\right) e ^ {-\frac{x ^2}{2}} dx = 0$, так как это интеграл от нечетной функции по симметричному промежутку. Тогда имеем:
\[
    \varphi_{\xi}\left(t\right) = \frac{1}{\sqrt{2 \pi}} \int_{-\infty}^{+\infty} \cos\left(tx\right) e ^ {-\frac{x ^2}{2}} dx
\]
Возьмем производную по $t$:
\[
    \varphi_{\xi}'\left(t\right) = -\frac{1}{\sqrt{2 \pi}} \int_{-\infty}^{+\infty} x\sin\left(tx\right) e ^ {-\frac{x ^ 2}{2}} dx
     = \frac{1}{\sqrt{2 \pi}} \int_{-\infty}^{+\infty} \sin\left(tx\right) d\left(e ^ {-\frac{x ^ 2}{2}}\right)=\]
\[
    = \frac{1}{\sqrt{2 \pi}} \sin\left(tx\right) e ^ {-\frac{x ^ 2}{2}} |_{-\infty}^{+\infty} - \frac{t}{\sqrt{2 \pi}} \int_{-\infty}^{+\infty} \cos\left(tx\right) e ^ {-\frac{x ^ 2}{2}} dx = 0 -\frac{t}{\sqrt{2 \pi}} \int_{-\infty}^{+\infty} \cos\left(tx\right) e ^ {-\frac{x ^ 2}{2}} dx = - t \varphi_{\xi}\left(t\right)
\]
Теперь надо решить дифференциальное уравнение: 
$$ \varphi_\xi'\left(t\right) = - t \varphi_{\xi}\left(t\right)$$
$$\frac{\varphi_\xi'\left(t\right)}{\varphi_\xi\left(t\right)} = -t$$
Интегрируем обе части:
$$ \int \frac{ d\left(\varphi_\xi\left(t\right)\right)}{\varphi_\xi\left(t\right)} = \ln \abs{\varphi_\xi\left(t\right)} + C = \int -t dt = -\frac{t ^ 2}{2}$$
$$\varphi_\xi\left(t\right) = C' e ^ {-\frac{t ^ 2}{2}}$$
$$\varphi_\xi\left(0\right) = 1 = C' \Rightarrow C' = 1$$
(Ответ никак нулевым быть не может, поэтому, когда мы поделили на хар функцию ничего плохого не произошло)
\end{example}
\begin{advice}
Пусть случайная величина $\xi$ обладает конечным $k$-тым моментом, то есть $ \mathbb{E}  \abs{\xi} ^ k < \infty$. Тогда $\varphi$ имеет непрерывную $k$-тую производную и
$
    \varphi^{(k)}\left(0\right) = i ^ k  \mathbb{E}  \xi ^ k
$
\end{advice}
\begin{proof}
Заметим, что:
\[
    \abs{\frac{e ^ {i \Delta t \xi} - 1}{\Delta t}} \leq \frac{\sqrt{\left(\cos\left(\Delta t\xi\right) - 1\right) ^ 2 + \sin ^ 2\left(\Delta t\xi\right)}}{\abs{\Delta t}} = \frac{\sqrt{2 - 2 \cos\left(\Delta t\xi\right)}}{\abs{\Delta t}} = \frac{2 \abs{\sin\left(\frac{\Delta t\xi}{2}\right)}}{\Delta t} \leq \frac{2 \cdot \frac{\abs{\Delta t\xi}}{2}}{\abs{\Delta t}} = \abs{\xi}
\]
Возьмем вместо $\Delta t$ последовательность $a_n \to 0$, Получим, что последовательность случайных величин, сходящихся почти наверное и ее предел ($i \xi$) ограничены случайной величиной ($\xi$)с конечным ожиданием, получаем по теорема Лебега: \[
    \mathbb{E}  \frac{e ^ {i a_n\xi} - 1}{a_n} \to  \mathbb{E}  i \xi
\]
Теперь посчитаем производную хар фукнции:
\[
    \varphi'\left(t\right) = \lim_{\Delta t \to 0} \frac{\varphi\left(t + \Delta t\right) - \varphi\left(t\right)}{\Delta t} = \lim_{\Delta t\to 0}  \mathbb{E}  e ^ {i t \xi} \cdot \frac{e ^ {i \Delta t \xi} - 1}{\Delta t} = i  \mathbb{E}  e ^ {it \xi} \xi
\]
Теперь очевидна непрерывность первой производной и $\varphi'\left(0\right) = i  \mathbb{E}  \xi$, для производных высших порядков аналогично

\end{proof}


\begin{theorem}
\[
    \xi_n \xrightarrow{\textit{d}} \xi \Leftrightarrow \forall t \lim_{n \to \infty} \varphi_{\xi_n}\left(t\right) = \varphi_{\xi}\left(t\right)
\]
\end{theorem}

\begin{proof}
$\Rightarrow$ \\
Очевидно по Теореме 1 \\
$\Leftarrow$ \\
Докажем при условиии $\sup_{n}  \mathbb{E}  \xi_n^2 \leq C < \infty$. По неравенству Чебышева:
\[
    P\left(\abs{\xi_n\geq A}\right) \leq \frac{ \mathbb{E}  \xi ^ 2}{A} \leq \frac{C}{A}
\]
Пусть $f \ -$ ограниченная непрерывная функция и $M = \sup \abs{f}$. Из записанного неравенства Чебышева следует, что $\forall \varepsilon > 0 \exists A$: 
\[
    P\left(\abs{\xi_n} \geq A\right) \leq \varepsilon, \ P\left(\abs{\xi} \geq A\right)  \leq \varepsilon
\]
Пусть непрерывная ограниченная $f_\varepsilon$ совпадает с $f$ на $[-A, A]$, потом от $-A-1$ до $-A$ и от $A + 1$ до $A$ она будет прямой из 0 в $f\left(-A\right)$ и $f\left(A\right)$ соответственно, а дальше будет повторять этот шаблон (периодическая).
Заметим, что:
\[
    \abs{ \mathbb{E}  f_{\varepsilon}\left(\xi_n\right) -  \mathbb{E}  f\left(\xi_n\right)} < 2M\varepsilon, \ \abs{ \mathbb{E}  f_{\varepsilon}\left(\xi\right) -  \mathbb{E}  f\left(\xi\right)} < 2M\varepsilon
\]
Теперь равномерно приблизим $f_{\varepsilon}$ комбинацией $\sin$, $\cos$ (знаем с матана, что периодическую можно так приблизить). А из сходимости хар функции мы знаем, что:
\[
   \lim_{n \to \infty}  \mathbb{E}  \sin\left(\xi_n\right) =  \mathbb{E}  \sin\left(\xi\right)\]\[
   \lim_{n \to \infty}  \mathbb{E}  \cos\left(\xi_n\right) =  \mathbb{E}  \cos\left(\xi\right)
\]
То есть получаем $\lim_{n \to \infty}  \mathbb{E}  f_{\varepsilon}\left(\xi_n\right) =  \mathbb{E}  f_{\varepsilon}\left(\xi\right)$. В итоге, вспоминая те неравенства с $2M \varepsilon$ и устремляя $n \to \infty$:
\[
     \mathbb{E}  f\left(\xi\right) - 4M \varepsilon \leq \inf \lim_{n \to \infty}  \mathbb{E}  f\left(\xi_n\right) \leq \sup \lim_{n \to \infty}  \mathbb{E}  f\left(\xi_n\right) \leq  \mathbb{E}  f\left(\xi\right) + 4M \varepsilon
\]
Устремляя $\varepsilon \to 0$, получаем $\lim_{n \to \infty}  \mathbb{E}  f\left(\xi_n\right) =  \mathbb{E}  f\left(\xi\right)$, что доказывает сходимость по распределению.
\end{proof}

\begin{corollary}
$\varphi_{\xi} \equiv \varphi_{\eta} \Rightarrow F_{\xi} \equiv F_\eta$
\end{corollary}
\begin{proof}
Предыдущая теорема + Теорема 1.
\end{proof}

\begin{theorem}(Центральная предельная теорема) \\
Пусть $\xi_n \ -$ последовательность независимых одинаково распределенных случайных величин, причем $ \mathbb{E}  \xi_1 = \mu$, $ \mathbb{D} \xi_1 = \sigma ^ 2$. Тогда $\forall x$:
\[
    \lim_{n \to \infty} P\left(\frac{\xi_1 + \cdots +\xi_n - n \mu}{\sigma \sqrt{n}} \leq t\right) = \frac{1}{\sqrt{2 \pi}} \int_{-\infty}^{t} e ^ {-\frac{x ^ 2}{2}}dx
\]
(справа записана $F\left(t\right) \ - $ функция распределения случайной величины с распределением $N\left(0, 1\right)$)

\end{theorem}
\begin{proof}
Переходя к случайным величинам $\xi_{i} = \frac{\xi_{i} - \mu}{\sigma}$ дальше будем считать, что $ \mathbb{E}  \xi_{i} = 0$ и $ \mathbb{D} \xi_{i} = 1$. Пусть $\varphi \ - $ хар функция случаной величины $\xi_1$. Тогда хар функция случаной величины 
\[
    \frac{\xi_1 + \cdots + \xi_n}{\sqrt{n}}
\]
равна 
\[
    \varphi_{n}\left(t\right) = \left(\varphi\left(\frac{t}{\sqrt{n}}\right)\right) ^ n
\]
Разложим $\varphi\left(\frac{t}{\sqrt{n}}\right)$ в ряд Тейлора в 0 (при $n \to \infty$), помним предложение 3:
\[
    \varphi\left(\frac{t}{\sqrt{n}}\right) = \varphi\left(0\right) + x\varphi'\left(0\right) + \cdots = 1 + 0 -\frac{t ^ 2}{2n} + o\left(\frac{1}{n}\right)
\]
Следовательно получаем:
\[
    \varphi_{n}\left(t\right) = \left(1 -\frac{t ^ 2}{2n} + o\left(\frac{1}{n}\right)\right) ^ n = \left(1 -\frac{t ^ 2}{2n} + o\left(\frac{1}{n}\right)\right) ^ {-\frac{2n}{t ^ 2} \cdot \left(-\frac{t ^ 2}{2}\right)} \xrightarrow{n \to \infty} e ^ {-\frac{t ^ 2}{2}}
\]
Получили характеристическую функцию нормального распределения, то есть и функции распределения должны совпадать.
\end{proof}


\clearpage

\section{Неравенство типа Хефдинга-Чернова}

\begin{theorem} (Неравенство Хефдинга-Чернова) \\
Пусть случайные величины $\xi_1, \cdots \xi_n$ независимы и $a_i \leq \xi_i \leq b_i$. Тогда для случайной величины $S_n = \xi_1 + \cdots + \xi_n$ и для каждого $t > 0$ выполнено
$$P\left(\abs{S_n -  \mathbb{E}  S_n} \geq t\right) \leq 2 \exp \left(-\frac{t ^ 2}{4 \sum_{i=1}^n \left(b_i - a_i\right) ^ 2}\right)$$
\end{theorem}
\begin{proof}
Пусть $\eta_i = \xi_i -  \mathbb{E}  \xi_i$, тогда $\abs{\eta_i} \leq b_i - a_i$. Заметим, что для каждого $\lambda > 0$ (просто домножили и взяли экспоненту):
$$P\left({S_n -  \mathbb{E}  S_n} \geq t\right) = P\left(\sum_{i = 1}^{n} \eta_i \geq t\right) = P\left(e ^ {\lambda \sum \eta_i} \geq e ^ {\lambda t}\right)$$
Теперь применим неравенство Маркова:
$$P\left(e ^ {\lambda \sum \eta_i} \geq e ^ {\lambda t}\right) \leq e ^ {-\lambda t}  \mathbb{E}  e ^ {\lambda \sum \eta_i}$$
Вспомним, что $\eta_1, \cdots \eta_n$ независимы:
$$ e ^ {-\lambda t}  \mathbb{E}  e ^ {\lambda \sum \eta_i} = e ^ {-\lambda t} \prod  \mathbb{E}  e ^ {\lambda \eta_i}$$
Оценим каждый множитель $ \mathbb{E}  e ^ {\lambda \eta_i}$ отдельно. Разложим его в ряд Тейлора:
$$ \mathbb{E}  e ^ {\lambda \eta_i} = 1 + \lambda  \mathbb{E}  \eta_i + \lambda ^ 2  \mathbb{E}  \eta_i ^ 2 + \sum_{k = 3}^{\infty} \frac{1}{k!}\lambda ^ k \eta ^ k \leq 1 + \lambda ^ 2 \left(b_i - a_i\right) ^ 2 + \sum_{k =3}^{\infty}\frac{1}{k!}\lambda ^ k \left(b_i - a_i\right) ^ k$$
Докажем, что при $R > 0$:
$$1 + \frac12 R ^ 2 + \sum_{k =3}^{\infty}\frac{1}{k!} R ^ k \leq e ^ {R ^ 2}$$
Если $R > 1$:
$$1 + \frac12 R ^ 2 + \sum_{k =3}^{\infty}\frac{1}{k!} R ^ k = 1 + \frac12 R ^ 2 + \sum_{k=2}^{\infty} \frac1{k!} R^{2k} \left(\frac{k!}{\left(2k-1\right)!} R ^ {-1} + \frac{n!}{\left(2n\right)!}\right) \leq 1 + \frac12 R ^ 2 + \sum_{k=2}^{\infty} \frac1{k!}R ^ {2k} = e ^ {R ^ 2}$$
Если же $R \leq 1$:
$$1 + \frac12 R ^ 2 + \sum_{k =3}^{\infty}\frac{1}{k!} R ^ k \leq 1 + \frac12 R ^2 + \sum_{k=3}^{\infty} \frac{1}{2 ^ {k - 1}} R ^ 2 = 1 + R ^ 2 \leq e ^ {R ^ 2}$$
Таким образом:
$$P\left({S_n -  \mathbb{E}  S_n} \geq t\right) \leq \exp \left(-\lambda t + \lambda ^2 \sum_{i=1}^{n}\left(b_i - a_i\right) ^ 2\right)$$
Взяв $\lambda = \frac{t}{2 \sum \left(a_i - b_i\right) ^ 2}$ получим желаемое неравенство. Для $P\left({S_n -  \mathbb{E}  S_n} \geq t\right)$ получим такую же оценку, и, объединяя их, получим оценку на модуль, только придется домножить оценку на 2.
\end{proof}

\begin{corollary}
Пусть $\xi_i \sim Bern\left(p\right) \ -$ набор $n$ независимых случайных величин, \\ $S_n = \xi_1 + \cdots + \xi_n$, тогда 
$$P\left(\abs{\frac{S_n}{n} - p} \geq t\right) \leq 2 e ^ {-\frac{n t ^ 2}{4}}$$ 
\end{corollary}
\begin{proof}
Разделим каждую случайную величину на $n$, тогда $ \mathbb{E}  \frac{S_n}{n} = p$, а $\sum_{i = 1}^{n} \left(a_i - b_i\right) ^ 2 = n \cdot \frac1{n ^ 2} = \frac1n$. Подставляем и получаем, нужное неравенство
\end{proof}

\begin{example}
Пусть в ящике какое-то количество черных и белых шаров. Каким должен быть размер выборки, чтобы оценить долю белых шаров с малой погрешностью? Пусть $\xi_i \ -$ бернуллевская случайная величина, равная 1, если шар белого цвета и 0, если цвет черный. Мы хотим оценить вероятность успеха $p$. По неравенству выше:
$$P\left(\abs{\frac{S_n}{n} - p} \geq t\right) \leq 2e^{-\frac{nt^ 2}{4}} \leq \varepsilon$$
Тогда при размере выборки $n = O\left(\frac{\ln \frac1\varepsilon}{t ^ 2}\right)$ выборочное среднее приближает реальную долю белых шаров с точностью $t$ с вероятностью более $1 - \varepsilon$(то есть вероятность, что наша оценка верна $ \geq 1 - \varepsilon$)
\end{example}

\clearpage

\section{Теоремы непрерывности}
Для применения ЦПТ на практике важную роль играют так называемые теоремы о непрерывности.

\begin{advice}
Если последовательность случайных величин $\xi_n \xrightarrow{\text{d}} \xi$, то для всякой непрерывной $g$ $g\left(\xi_n\right) \xrightarrow{\text{d}} g\left(\xi\right)$
\end{advice}
\begin{proof}
Следует из Теоремы 1(эквивалентное определение сходимости по распределению).
\end{proof}

\begin{lemma}
Пусть $X, Y, Z \ - $ случайные величины. Тогда 
$$P\left(X + Y \leq t\right) \leq P\left(X + Z \leq t + \varepsilon\right) + P\left(\abs{Y - Z} \geq \varepsilon \right), \ \forall t \in \mathbb{R}, \forall \varepsilon > 0$$
\end{lemma}
\begin{proof}
Заметим, что
$$P\left(X + Y \leq t\right) \leq  P\left(X + Y \leq t, \ \abs{Y - Z} \leq \varepsilon\right) + P\left(X + Y \leq t, \ \abs{Y - Z} \geq \varepsilon \right) \leq$$
$$ \leq  P\left(X + Z - \varepsilon \leq t \right) + P\left(\abs{Y - Z} \geq \varepsilon\right)$$
\end{proof}

\begin{advice}
Если $\xi_{n} \xrightarrow{\text{P}} a = const$ и $\eta_n \xrightarrow{\text{d}} \eta$, то $\xi_n \eta_n \xrightarrow{\text{d}} a \eta$,  $\xi_n + \eta_n \xrightarrow{\text{d}} a + \eta$
\end{advice}
\begin{proof}
Докажем утверждение для суммы. Пусть $\varepsilon > 0$, тогда по лемме имеем:
$$P\left(\xi_n + \eta_n \leq t\right) \leq F_{\eta_n}\left(t - a + \varepsilon\right) + P\left(\abs{\xi_n- a} \geq \varepsilon\right)$$
и
$$P\left(\xi_n + \eta_n \leq t\right) \geq F_{\eta_n}\left(t - a - \varepsilon\right) - P\left(\abs{\xi_n- a} \geq \varepsilon\right)$$
Устремляя сначала $n \to \infty$, а затем $\varepsilon \to 0$. Из сходимости по вероятности получаем, что $P\left( \abs{\xi_n - a} \geq \varepsilon\right) \to 0$. Тогда в итоге имеем:
    $$\lim \left(F_{\eta_n}\left(t - a - \varepsilon\right) - P\left(\abs{\xi_n- a} \geq \varepsilon\right)\right) \leq \lim F_{\xi_n + \eta_n}\left(t\right) \leq \lim \left(F_{\eta_n}\left(t - a + \varepsilon\right) + P\left(\abs{\xi_n- a} \geq \varepsilon\right)\right)$$
    $$\lim F_{\xi_n + \eta_n}\left(t\right) = F_{\eta}\left(t - a\right) = F_{a + \eta}\left(t\right)$$
Теперь докажем утверждение для произведения. Пусть $a = 0$ (случай, когда $a \neq 0$ выводится из суммы 
$\left(\xi_n - a\right)\eta_n$ и $a \eta_n$). Теперь заметим, что $\forall \varepsilon > 0 \forall C > 0$
верно включение:
 $$\left\{\abs{\xi_n \eta_n} > \varepsilon\right\} \subseteq \left\{\abs{\eta_n} > C\right\} \bigcup \left\{\abs{\xi_n} > \frac{\varepsilon}{C}\right\}$$
 (Пояснение: это верно, так как пересечение отрицания обоих событий точно приводит к противоречию). Тогда, переходя к вероятностям, получаем:
 $$P\left(\abs{\xi_n \eta_n} > \varepsilon\right) \leq 1 - F_{\eta_n}\left(C\right) + F_{\eta_n}\left(-C\right) + P\left(\abs{\xi_n}> \frac\varepsilon C\right)$$
Устремляя сначала $n \to 0$, а затем $C \to \infty$, получаем, что $\xi_n \eta_n \xrightarrow{\text{P}} 0$ $\Rightarrow$ $\xi_n \eta_n \xrightarrow{\text{d}} 0$. 
\end{proof}

\begin{example}(Выборочная дисперсия)\\
Пусть задана последовательность независимых и одинаково распределенных случайных величин $\xi_i$, причем $ \mathbb{E}  \xi_i = \mu$ и $ \mathbb{D} \xi_i = \sigma^ 2$. Тогда последовательность случайных величин
$$s_n ^ 2 = \frac1{n-1} \sum_{i = 1}^{\infty} \left(\xi_i - \overline{\xi_n}\right) ^ 2 \xrightarrow{\text{P}} \sigma ^ 2$$
где $\overline{\xi_n} = \frac{\xi_1 + \cdots+ \xi_n}{n}$ (умножаем на $\frac1{n-1}$, а не на $\frac1n$, чтобы $ \mathbb{E}  s_n ^ 2 = \sigma ^ 2$, то есть таким образом посчитанная диспресия по вероятности сходится к именно тому, чему и надо, в другом случае будет небольшое смещение в $\frac{n - 1}{n}$ раз, но с увеличение $n$ разница в любом случае будет стрираться). Действительно:
$$s_n ^ 2 = \frac1{n - 1} \left(\sum_{i = 1}^{\infty} \xi_i ^ 2 - 2 n\overline{\xi_n} \cdot \frac1n \sum_{i = 1}^{\infty} + n \overline{\xi_n} ^ 2\right) = \frac1{n - 1} \left(\sum_{i = 1}^{\infty} \xi_i ^ 2 - n \overline{\xi_n} ^ 2\right) = \frac{n}{n - 1} \cdot \frac1n \sum_{i = 1}^{\infty} \xi_i ^ 2 - \frac{n}{n - 1} \overline{\xi_n} ^ 2$$
Теперь заметим, что по ЗБЧ:
$$\frac1n \sum_{i = 1}^{\infty} \xi_i ^ 2 \xrightarrow{\text{P}} \sigma ^ 2 + \mu, \ \overline{\xi_{n}} \xrightarrow{\text{P}} \mu$$
Получили искомую сходимость
\end{example}

\begin{example}
Пусть задана последовательность независимых и одинаково распределенных случайных величин $\xi_i$, причем $ \mathbb{E}  \xi_i = \mu$ и $ \mathbb{D} \xi_i = \sigma ^ 2 > 0$. Тогда из ЦПТ следует, что:
$$\frac{\xi_1 + \cdots + \xi_n- n \mu}{\sigma \sqrt{n}} = \frac{\sqrt{n}\left(\overline{\xi_n} - \mu\right)}{\sigma} \xrightarrow{\text{d}} \xi \sim N\left(0, 1\right)$$
Более того, так как $s_n ^ 2 \xrightarrow{\text{P}} \sigma ^ 2 > 0$, то имеет место сходимость по распределению величин:
$$\frac{\sqrt{n}\left(\overline{\xi_n} - \mu\right)}{s_n ^ 2} \xrightarrow{\text{d}} \xi \sim N\left(0, 1\right)$$

\end{example}

\begin{advice}
Пусть $a, h_n \in \mathbb{R}$, $\lim_{n \to \infty} h_n = 0$ и $f \ -$ непрерывно дифференцируемая функция на $\mathbb{R}$.
Если последовательность случайных величин $\xi_n$ сходится по распределению к $\xi$, то:
\[
    \frac{f\left(a + h_n \xi_n\right) - f\left(a\right)}{h_n} \xrightarrow{d} f'\left(a\right) \xi
\]
\end{advice}

\begin{proof}
Имеет место равенство:
\[
    \frac{f\left(a + h_n \xi_n\right) - f\left(a\right)}{h_n} =
    \frac{f\left(a + 1\cdot h_n \xi_n\right) - f\left(a + 0 \cdot h_n \xi_n\right)}{h_n} = 
    \frac1{h_n} \int_{0}^{1} f\left(a + t h_n \xi_n\right) d\left(a + t h_n \xi_n\right) =
    \]
    \[
    = \xi_n \int_0^1 f\left(a + t h_n \xi_n\right) dt
\]
Из предложения 5(самый конец) получаем, что $h_n \xi_n \xrightarrow{P} 0$.
Также заметим, что функция
\[
    g\left(y\right) = \int_0^1 f(a + ty) dt
\]
непрерывна. Следовательно по предложению 4:
\[
    g\left(h_n \xi_n\right) = \int_0^1 f\left(a + t h_n \xi_n\right) dt \xrightarrow{P} g(0) = f'\left(a\right)
\]
Теперь снова используя предложение 5, получаем нужную сходимость.
\end{proof}

\begin{example}
Пусть задана последоватеность независисмых и одинаково распределенных случайных величин $\xi_i$, причем 
$ \mathbb{E}  \xi_i = \mu$ и $ \mathbb{D} \xi_i = \sigma ^ 2 > 0$. Если $h \ - $ непрерывно дифференцируемая функция, то
\[
    \sqrt{n}  \left(h\left(\overline{\xi_n}\right) - h\left(\mu\right)\right) \xrightarrow{d} \xi \sim N\left(0, q ^ 2\right), \ q = \sigma h'\left(\mu\right)
\]
Действительно, имеем равенство
\[
    \sqrt{n} \frac{\left(h\left(\overline{\xi_n}\right) - h\left(\mu\right)\right)}{\sigma} = 
    \frac1{\sigma} \cdot \frac{h\left(\mu + n ^ {-\frac12} \sigma \zeta_n\right) - h\left(\mu\right)}{n ^ {-\frac12}}
\]
где(сходимость по ЦПТ)
\[
    \zeta_n = \frac{\sqrt{n} \left(\overline{\xi_n}- \mu\right)}{\sigma} \xrightarrow{d} \xi \sim N\left(0, 1\right)
\]
Используем предложение 6 и получаем требуемое.
\end{example}

\begin{example}
Пусть $\xi_1, \cdots \xi_n$ положительные независимые одинаково распределенные случайные величины, 
$ \mathbb{E}  \xi_1 = \mu$, $0 <  \mathbb{D} \xi_1 = \sigma ^ 2 < \infty$.
Рассмотрим случайную величину $S_n = \xi_1 + \cdots + \xi_n$ и найдем предел в смысле сходимости по распределению у последовательности случайных величин
$\sqrt{n} \left(\frac{n}{S_n} - \frac1\mu\right)$.\\
Первый способ:
\[
    \sqrt{n} \left(\frac{n}{S_n} - \frac1\mu\right) = -\frac1\mu \frac{n}{S_n} \sqrt{n} \left(\frac{S_n}{n} - \mu\right)
\]
По ЦПТ
\[
    \sqrt{n} \left(\frac{S_n}{n} - \mu\right) \xrightarrow{d} \xi \sim N\left(0, \sigma ^ 2\right)
\]
По ЗБЧ
\[
    \frac{n}{S_n} \xrightarrow{P} \frac1\mu
\]
Таким образом имеем:
\[
    \sqrt{n} \left(\frac{n}{S_n} - \frac1\mu\right) \xrightarrow{d} -\frac1\mu ^2  \xi \sim N\left(0, \frac{\sigma ^ 2}{\mu ^ 4}\right)
\]
Второй способ: \\
Пусть $h\left(x\right) = \frac1x$, тогда 
\[
    \sqrt{n} \left(\frac{n}{S_n} - \frac1\mu\right) = \sqrt{n} \left(h\left(\frac{S_n}{n}\right) - h\left(m\right)\right)
    \xrightarrow{d} \xi \sim N\left(0, \sigma ^2 \left(h'\left(\mu\right)\right) ^ 2\right) = N\left(0, \frac{\sigma ^ 2}{\mu ^ 4}\right)
\]

\end{example}

\clearpage

\section{Многомерная характеристическая функция и ЦПТ}

\begin{definition}
$\textit{Характеристическая функция случайного вектора}$ $\xi = \left(\xi_1, \cdots, \xi_m\right) ^ T$ определяется равенством
\[
    \varphi_{\xi}\left(x\right) =  \mathbb{E}  e ^ {i  x \xi } = \mathbb{E} e ^ {i \sum_{i = 1}^{m} x_i \xi_i}
\]

\end{definition}

\begin{advice}
$\varphi_{\xi} \equiv \varphi_\eta$ $\Leftrightarrow$ $\xi$ и $\eta$ имеют одинаковые распределения
\end{advice}
\begin{proof}
Заметим, что
\[
    F_{\xi}\left(x_1, \cdots, x_m\right) =  \mathbb{E}  I_{\leq x_1}\left(\xi_1\right) \cdots I_{\leq x_m}\left(\xi_m\right)
\]
По аналогии с одномерным случаем, нам достаточно доказать, что 
\[
     \mathbb{E}  g_1 \left(\xi_1\right) \cdots g_m\left(\xi_m\right) =  \mathbb{E}  g_1 \left(\eta_1\right) \cdots g_m \left(\eta_m\right)
\]
для непрерывных периодических функций $g_k\left(u\right)$. Такие функции приближаются линейными комбинациями функций вида $e ^ {i \mu_k u}$. Значит, достаточно проверять совпадение выражений вида
\[
     \mathbb{E}  \exp \left(i \mu_1 \xi_1 + \cdots + i \mu_m \xi_m\right) =  \mathbb{E}  \exp \left(i \mu_1 \eta_1 + \cdots + i \mu_m \eta_m\right)
\]
А это у нас есть(это хар функции).
\end{proof}

\begin{corollary}
Случайные величины $\xi_1, \cdots, \xi_m$ независимы тогда и только тогда, когда
\[
    \varphi_{\xi}\left(y_1, \cdots , y_m\right) = \varphi_{\xi_1}\left(y_1\right) \cdots \varphi_{\xi_m}\left(y_m\right)
\]
\end{corollary}
\begin{proof}
$\Rightarrow$
\[
    \varphi_{\xi}(y_1,\cdots,y_m) = \mathbb{E}e^{i(\xi_1 y_1+\cdots+\xi_m y_m)} = \mathbb{E}e^{i
    \xi_1 y_1} \cdots e^{i\xi_m y_m} =
\]
В силу независимости $\xi_i$
\[
    = \mathbb{E}e^{i
    \xi_1 y_1} \cdots \mathbb{E}e^{i\xi_m y_m} =  \varphi_{\xi_1}\left(y_1\right) \cdots \varphi_{\xi_m}\left(y_m\right)
\]
$\Leftarrow$\\
    Сделаем новый вектор $(\eta_1, \cdots , \eta_m)$, так что:\\
    \begin{enumerate}
        \item Распр $\eta_i$ = распр $\xi_i$ \text{ }$\forall i \in \{1, \cdots , m\}$
        \item $\eta_1, \cdots, \eta_m$  независимы
    \end{enumerate}
    Определим
    $F(y) = F_{\eta_1}(y_1) \cdots F_{\eta_m}(y_m)$. Тогда мы знаем, что существует вектор, у которого такая функция распределения, из чего непременно следует независимость $\eta_1, \cdots, \eta_m$\\
    Посчитаем хар. функцию $\eta = (\eta_1, \cdots, \eta_m)$
    \[
        \varphi_{\eta}(y) = 
        \left[\begin{array}{ll}
        \text{в силу}\\
        \text{нез - сти $\eta_i$}\\
        \end{array}
        \right] = \varphi_{\eta_1}(y_1) \cdots \varphi_{\eta_m}(y_m) = \varphi_{\xi_1}(y_1) \cdots \varphi_{\xi_m}(y_m) = \varphi_{\xi}(y)
    \]
    По предыдущему предложению и независимости $\eta_1, \cdots, \eta_m$ $\Rightarrow$ $\xi_1, \cdots, \xi_m$ независимы. 
\end{proof}

\begin{theorem}
Пусть последовательность независимых одинаково распределенных случайных векторов $\xi^n = \left(\xi^n_1, \cdots, \xi^n_m\right) \in \mathbb{E}$ имеют конечные
\[
    \mathbb{E} \xi_i^1 = \mu_i, \ r_{ij} = \text{cov} \left(\xi_i^1, \xi_j^1\right)
\]
Тогда величины
\[
    \eta_i^n = \frac{\xi_i^1+ \cdots + \xi_i^n - n \mu_i}{\sqrt{n}}
\]
таковы, что последовательность векторов $\eta^n = \left(\eta_1^n, \cdots,  \eta_m^n\right) \xrightarrow{d} \eta$, характеристическая функция которого имеет вид
\[
    \varphi_{\eta}\left(y\right) = \exp \left(-\frac{ \left<yR, y \right>}{2}\right), \ R = \left(r_{ij}\right)
\]
\end{theorem}

\begin{proof}
В векторной форме:
\[  
    \eta^{n} = \dfrac{\sum\limits_{s=1}^{n}\xi^{s} - \mu n}{\sqrt{n}}
\]
Запишем хар. функцию:
\[
    \varphi_{\eta^{n}}(t) = \varphi_{(\xi^{1} - \mu + \cdots + \xi^{n} - \mu)/\sqrt{n}}(t) = \varphi_{\xi^{1} - \mu + \cdots + \xi^{n} - \mu}\left(\frac{t}{\sqrt{n}}\right) = 
    \]
    В силу независимости $\xi_i$ и их одинаковой распределенности
    \[
        = \left(\varphi_{\xi^{1} - \mu}\left(\frac{t}{\sqrt{n}}\right)\right)^n
    \]
Разложим в ряд тейлора:
\[
    \left(\varphi_{\xi^{1} - \mu}\left(\frac{t}{\sqrt{n}}\right)\right)^n = \left(1 + \bigtriangledown \varphi_{\xi^{1} - \mu}(0) \cdot \frac{t}{\sqrt{n}} + \frac{1}{2}\cdot D^2\varphi_{\xi^{1} - \mu}(0) \left<\frac{t}{\sqrt{n}}, \frac{t}{\sqrt{n}}\right>  + o\left(\frac1n\right)\right)^n
\]
Рассмотрим на 2-мерном случае (на других аналогично):
\[
    \varphi_{\xi}(t_1, t_2) = \mathbb{E}e^{it_1\xi_1 + it_2 \xi_2}
\]
Первая производная:
\[
    \frac{d}{d t_{j}} \varphi_{\xi}(t) = i\mathbb{E}\xi_{j} e^{it_1\xi_1 + it_2 \xi_2}; \text{ } \frac{d}{d t_{j}} \varphi_{\xi}(0) = i\mathbb{E}\xi_{j}
\]
В нашем случае нетрудно понять, что $\bigtriangledown \varphi_{\xi^{1} - \mu}(0) = 0$, так как у нас $\xi$ это $\xi^{1} - \mu$, а $i\mathbb{E}(\xi_{i}^{1} - \mu_{i}) = 0$\\
Вторая производная:
\[
    \dfrac{d^2}{d t_{j} t_{s}} \varphi_{\xi}(t) = -\mathbb{E}\xi_{j} \xi_{s} e^{it_1\xi_1 + it_2 \xi_2}; \text{ } \frac{d^2}{d t_{j} t_{s}} \varphi_{\xi}(0) = -\mathbb{E}\xi_{j}\xi_{s}
\]
В нашем случае $-\mathbb{E}(\xi_{j}^{1} - \mu_{j})(\xi_{s}^{1} - \mu_{s}) = -r_{js}$\\
Получаем:
\[
    \left(1 - \dfrac{1}{2}\langle Rt, t \rangle \frac{1}{n} + o\left(\frac1n\right)\right)^n \xrightarrow{n \to \infty} e^{-\frac{1}{2}\langle Rt, t \rangle}
\]
\end{proof}

\clearpage

\section{Многомерное нормальное распределение}  

\begin{definition} Случайный вектор $\xi \in \text{Mat}_{m \times 1}$ имеет $\textit{нормальное распределение}$ или является $\textit{гауссовским}$, если $\forall x \in \mathbb{R} ^ m$
\[
    \varphi_{\xi}\left(x\right) = \mathbb{E}  e ^ {i  x \xi} = e ^ {i x\mu - \frac12 x R x ^ T}
\]
где $\mu = \left(\mu_1, \cdots , \mu_m\right)^T$, $R \in \text{Mat}_{m \times m}\ - $ симметричная неотрицательно определенная. Далее кратко пишем
\[
    \xi \sim N\left(\mu, R\right)
\]
\end{definition}

\begin{definition}
Пусть $\xi \in \text{Mat}_{m \times 1}$ случайный вектор. Матрица $R \in \text{Mat}_{m \times m}$ с компонентами $r_{ij} = \text{cov} \left(\xi_i, \xi_j\right)$
называется $\textit{ковариационной матрицей}$ вектора $\xi$. Можно еще написать, что 
\[
    R = \text{cov} \left(\xi, \xi\right) = \mathbb{E} \left(\xi - \mathbb{E} \xi\right) \left(\xi - \mathbb{E} \xi\right)^T
\]
\end{definition}
\begin{lemma}
Симметричная неотрицательно определенная матрица $R \in \text{Mat}_{m \times m}$ является ковариционной матрицей случайного вектора-столбца $\xi \in \text{Mat}_{m \times 1}$ тогда и только тогда, когда $\forall x, y \in \mathbb{R} ^ m$
\[
    \text{cov} \left(x\xi, y \xi\right) = x \text{cov} \left(\xi, \xi\right) y ^ T = x R y ^ T
\]
\end{lemma}

\begin{proof}
$\Rightarrow$ \\
Распишем по определению ковариации двух случайных величин (у нас именно они):
\[
    \text{cov} \left(x\xi, y \xi\right) = \mathbb{E} \left[\left(x \xi - \mathbb{E} x \xi\right) \left(y \xi - \mathbb{E} y \xi \right)\right] = \mathbb{E} \left[x \left(\xi - \mathbb{E} \xi\right) y \left(\xi - \mathbb{E} \xi\right)\right]
     = 
\]
Транспонируя скаляр, получаем тот же скаляр:
\[
    = \mathbb{E} \left[x \left(\xi - \mathbb{E} \xi\right) \left(y \left(\xi - \mathbb{E} \xi\right)\right) ^ T\right] 
    = \mathbb{E} \left[x \left(\xi - \mathbb{E} \xi\right)\left(\xi - \mathbb{E} \xi\right) ^ T y ^ T\right] 
    = x \mathbb{E} \left[\left(\xi - \mathbb{E} \xi\right)\left(\xi - \mathbb{E} \xi\right) ^ T\right] y ^ T
    = x \text{cov} \left(\xi, \xi\right) y ^ T
\]

$\Leftarrow$ \\
Возьмем $x = e_i$, $y = e_j$ (базисные единичные вектора). Тогда из данного равенства получим:
\[
    \text{cov} \left(e_i \xi, e_j \xi\right) = \text{cov} \left(\xi_i, \xi_j\right) = e_i R e_j = r_{ij}
\]
Следовательно $R = \text{cov} \left(\xi, \xi\right)$ по определению.

\end{proof}

\begin{advice}
$\forall A \in \text{Mat}_{m \times m}$ $\forall b \in \text{Mat}_{m \times 1}$ и $\xi \in \text{Mat}_{m \times 1} \ - $  случайного вектора, верно:
\[
    \text{cov} \left(A\xi + b, A \xi + b\right) = A R_\xi A ^ T
\]

\end{advice}
\begin{proof}
Распишем по определению: 
\[
    \text{cov} \left(A\xi + b, A \xi + b\right) = \mathbb{E} \left[\left(A \xi + b - \mathbb{E} \left(A\xi + b\right)\right) \left(A \xi + b - \mathbb{E} \left(A\xi +b\right)\right) ^ T\right] = 
\]
\[
    = \mathbb{E} \left[\left(A \xi + b - b - \mathbb{E} A\xi\right) \left(A \xi + b -b -  \mathbb{E} A\xi\right) ^ T\right] = \mathbb{E} \left[\left(A \xi  - \mathbb{E} A\xi\right) \left(A \xi  -  \mathbb{E} A\xi\right) ^ T\right] = 
    \]\[
    = A \mathbb{E} \left[\left(\xi  - \mathbb{E}\xi\right) \left(\xi  -  \mathbb{E}\xi\right) ^ T\right] A ^ T
    = A R_\xi A ^ T
\]
\end{proof}
\begin{corollary}
Если вектор $\xi \sim N\left(\mu, R\right)$, то вектор $A\xi + b \sim N\left(A\mu + b, A R_\xi A ^ T\right)$
\end{corollary}

\begin{theorem}
Вектор $\xi \in \text{Mat}_{m \times 1}$ имеет нормальное распределение тогда и только тогда, когда $\forall x \in \mathbb{R} ^ m$ случайная величина $ x\xi$ имеет нормальное распределение
\end{theorem}
\begin{proof}
$\Rightarrow$ \\
Если $\xi$ нормальный вектор, то 
\[
    \varphi_{x \xi} \left(t\right) = \mathbb{E} e ^ {it  x \xi } 
     = \varphi_{\xi}\left(tx\right) = \exp\left(-\frac12  txR \left(tx\right) ^ T + i  tx\mu\right) = 
\]
\[
    = \exp\left(-\frac12 t ^ 2 xR x^ T + it x \mu\right)
\]
Получили хар функцию нормального распределения $ x \xi \sim N\left(x \mu,  xR x^ T\right)$.\\
$\Leftarrow$ \\
В обратную сторону:
\[
    \varphi_{\xi}\left(x\right) = \mathbb{E} e ^ {i  x \xi } = \varphi_{ x \xi }\left(1\right) = \exp\left(-\frac12 \mathbb{D}  x \xi  + i\mathbb{E}  x \xi \right) = \exp\left(-\frac12  \text{cov} \left( x \xi ,  x \xi \right) + i  x \mu \right) = 
    \]\[
    = \exp\left(-\frac12   xR x^ T + i  x \mu \right)
\]
, где $R = \text{cov} \left(\xi, \xi\right)$, $\mu = \mathbb{E} \xi$. Последний переход вытекает из леммы2.
\end{proof}

\begin{corollary}
Если $\xi \sim N\left(\mu, R\right)$, то $R = \text{cov}\left(\xi, \xi\right)$, $\mu = \mathbb{E} \xi$.
\end{corollary}

\begin{corollary}
Если вектор $\xi = \left(\xi_1, \xi_2\right)$ имеет нормальное распределение и $\text{cov}\left(\xi_1, \xi_2\right) = 0$, то случайные величины $\xi_1$ и $\xi_2$ независимы.
\end{corollary}

\begin{proof}
Пусть 
\[
    \mu = \mathbb{E} \xi = \left(\mu_1, \mu_2\right)
\]
Заметим, что 
\[R = 
    \text{cov} \left(\xi, \xi\right)   = 
    \begin{pmatrix}
    \text{cov}\left(\xi_1, \xi_1\right) & 0\\
    0 & \text{cov}\left(\xi_2, \xi_2\right)
    \end{pmatrix}
\]
Теперь запишем харфункцию $\xi$:
\[
    \varphi_{\xi}\left(x_1, x_2\right) = \exp\left(-\frac 1 2 \left(x_1 ^ 2 \mathbb{D} \xi_1 + x_2 ^ 2 \mathbb{D} \xi_2 \right) + i \left(x_1 \mu_1 + x_2 \mu_2\right)\right) =
\]
\[
    = \exp\left(-\frac 1 2 x_1 ^ 2 \mathbb{D} \xi_1 + ix_1 \mu_1\right) \cdot \exp\left(-\frac 1 2 x_2 ^ 2 \mathbb{D} \xi_2 + ix_2 \mu_2\right)= \varphi_{\xi_1}\left(x_1\right) \varphi_{\xi_2}\left(x_2 \right)
\]
Теперь из следствия4 вытекает независимость $\xi_1$ и $\xi_2$
\end{proof}

\begin{corollary}
Если $\xi \sim N\left(\mu, R\right)$($\in \text{Mat}_{m \times 1}$), то $\exists A \in \text{Mat}_{m \times k}$, что $\xi = A \eta + \mu$, где $\eta = \left(\eta_1, \cdots , \eta_k\right) ^ T$, $\eta_i \ -$ независимые $N\left(0, 1\right)$ случайные величины. Причем $A A ^ T = R$
\end{corollary}
\begin{proof}
Пусть \[
    \xi' = \xi - \mu = \left(\xi_1', \cdots, \xi_m'\right)^ T
\]
Свели задачу к задачи нахождения ортонормированного базиса $\eta = \left(\eta_1, \cdots, \eta_k\right) ^ T$ в подпространстве $\langle \xi_1', \cdots, \xi_m' \rangle$ со скалярным произведением $\left(X, Y\right) = \mathbb{E} XY$. Эта задача решается методом Грама-Шмидта. Получили матрицу перехода $A$, что 
\[
    \xi - \mu = \xi' = A \eta
\]
То есть 
\[
    \xi = A \eta + \mu
\]
Осталось пояснить $A A ^ T = R$:
\[
    R = \text{cov} \left(\xi, \xi\right) = \text{cov} \left(\xi - \mu, \xi - \mu\right) = \text{cov} \left(A \eta, A \eta\right) = A \text{cov}\left(\eta, \eta\right) A^ T= A E A ^ T = A A^T
\]
$\text{cov}\left(\eta, \eta\right) = E$, так как это ортонормированный базис.
\end{proof}

\begin{theorem}
Если $\xi \sim N\left(\mu, R\right)$ (в этой теореме сделаем $\mu := \mu ^ T \in R ^ m$) и $\text{det}R \neq 0$, случайный вектор $\xi$ имеет плотность
\[
    \rho\left(x\right) = \frac1 {\left(2 \pi\right) ^ {\frac m 2} \sqrt{\text{det} R}}e ^ {-2 ^ {-1} \left(x - \mu\right) R ^ {-1} \left(x - \mu\right) ^ T}
\]
\end{theorem}

\begin{proof}
Так как $\xi = A \eta + \mu$, причем $\exists A ^ {-1}$, то
\[
    P\left(\xi \in B\right) = P\left(A \eta + \mu \in B\right) = \frac{1}{\left(\sqrt{2 \pi}\right) ^ m}
    \int_{A \eta + \mu \in B} e ^ {-2 ^ {-1}x x ^ T} dx = 
\]
\[
   =  \frac 1 {\left(2 \pi\right) ^ {\frac m 2}} \int_{B} e ^ {- 2 ^ {-1} \left(A ^ {-1}\left(x - \mu\right)\right)\left(A ^ {-1}\left(x - \mu\right)\right) ^ T} d\left(A ^ {-1} \left(x - \mu\right)\right)
   = \frac 1 {\left(2 \pi\right) ^ {\frac m 2} \text{det} A} \int_{B} e ^ {- 2 ^ {-1} \left(A ^ {-1}\left(x - \mu\right)\right)\left(A ^ {-1}\left(x - \mu\right)\right) ^ T} dx
\]

Остается заметить, что 
\[
    \text{det} A A ^ T = \left(\text{det} A\right) ^ 2 = \text{det} R
\]
и что
\[
    A ^ {-1}\left(x - \mu\right)\left(A ^ {-1}\left(x - \mu\right)\right) ^ T = 
    \left(x - \mu\right) A ^ {-1} \left(A ^ {-1}\right) ^ T \left(x - \mu\right) ^ T= 
    \]\[
    = \left(x - \mu\right) \left(A A ^ T\right) ^ {-1} \left(x - \mu\right) ^ T 
    = \left(x - \mu\right) R ^ {-1} \left(x - \mu\right) ^ T
\]

\end{proof}


\begin{example}
Пусть $\xi = \left(\xi_1, \cdots, \xi_n\right) ^ T$, где $\xi_i \sim N\left(0, \sigma ^ 2\right)$ и независимы между собой (или, что то же самое $\xi \sim N\left(0, \sigma ^ 2 E\right) $). Положим
\[
    \overline{\xi} = \frac{\xi_1 + \cdots + \xi_n}{n}, \ \zeta = \left(\xi_1 - \overline{\xi}\right) ^ 2
     + \cdots + \left(\xi_n - \overline{\xi}\right) ^ 2
\]
($\zeta \ - $ выборочная диспресия). Покажем, что $\overline{\xi}$ и $\zeta$ независимы. \\
Пусть $U \in \text{Mat}_{n \times n} \ - $ ортогональная матрица ($U U ^ T = E$), первая строка которой имеет вид $\left(n ^ {-\frac 1 2}, \cdots, n ^ {-\frac 1 2}\right)$. Тогда координаты вектора $u = U \xi \sim N\left(0, U \sigma ^ 2 E U ^ T\right) = N\left(0, \sigma ^ 2 E\right)$ являются независимыми. Заметим, что $u_{n}= \overline{\xi}\sqrt{n}$ и что
\[
    u ^ T u = u_1 ^ 2 + \cdots + u_n ^ 2 = n \overline{\xi ^ 2} + u_2 ^ 2 + \cdots + u_n ^ 2= \xi ^ T U ^ T U \xi = \xi ^ T \xi = \xi_1 ^ 2 + \cdots + \xi_n ^ 2
\]
Иначе говоря
\[
    u_2 ^ 2 + \cdots + u_n ^ 2 = \xi_1 ^ 2 + \cdots + \xi_n ^ 2 - n \overline{\xi ^ 2} 
\]
Теперь заметим
\[
    \zeta = \xi_1 ^ 2 + \cdots + \xi_n ^ 2 - 2 \sum_{i = 1}^{n} \xi_i \overline{\xi} + n \overline{\xi ^ 2}
    = \xi_1 ^ 2 + \cdots + \xi_n ^ 2 - n \overline{\xi ^ 2} = u_2 ^ 2 + \cdots + u_n ^ 2
\]
Так как $u_1, \cdots , u_n \ -$ независимы, то и $\frac{u_1}{\sqrt{n}} = \overline{\xi}$ и $u_2 ^ 2 + \cdots + u_n ^ 2 = \zeta$ тоже независимы. \\
Распределение величины $\chi = \eta_1^2 + \cdots + \eta_n^2$, где $\eta_i$ независимые c распределением $N\left(0, 1\right)$, называют распределением хи-квадрат с $n$ степенями свободы и обозначают через $\chi_{n} ^ 2$. 
Найдем плотсноть распределения $\chi_n ^ 2$:
\[
    P\left(\chi \leq t\right) = \left(2 \pi\right) ^ {-\frac{n}{2}} \int_{\eta_1 ^ 2 + \cdots + \eta_n ^ 2 \leq t} e ^ {-\frac{x_1 ^ 2 + \cdots + x_n ^ 2}{2}} dx = 
 \]
 Делаем сферическую замену ($w_n \ -$ площадь $n$-мерной единичной сферы):
 \[
     = \left(2 \pi\right) ^ {-\frac{n}{2}} w_n \int_{0}^{\sqrt{t}} r ^ {n - 1} e ^ {-\frac{ r ^ 2} 2} dr
 \]
 Тогда 
 \[
    \rho\left(t\right) = \left(2 \pi\right) ^ {-\frac{n}{2}} w_n \cdot \frac{1}{2} t ^ {-\frac{1}{2}} t ^ {\frac{n - 1}{2}} e ^ {-\frac{ t} 2} =  \frac{1}{2}  \left(2 \pi\right) ^ {-\frac{n}{2}} w_n  t ^ {\frac{n - 2}{2}} e ^ {-\frac{t ^ 2}{2}} \text{Ind}_{t > 0}
 \]
\end{example}

\clearpage
 \section{Условные математические ожидания: дискретный случай}

Предположим, что задана дискретная случайная величина
\[
    \xi\left(w\right) = \sum_{i = 1}^{n} x_i \text{Ind}_{A_i}\left(w\right)
\]
Рассмотрим следующую задачу: найти математическое ожидание $\xi$, если достоверно известно, что произошло событие $B$, $P\left(B\right) > 0$. Поскольку мы знаем, что событие $B$ произошло, то надо пересчитать вероятности $A_k$ с учетом новой информации, а именно, заменить $P\left(A_k\right)$ на $P\left(A_k | B\right)$. Таким образом, надо вычислить математическое ожидание не относительно исходной вероятностной меры $P$, а относительно условной вероятности $P\left(\cdot | B\right)$. 
\begin{definition}
Имеем:
\[
    \mathbb{E}\left(\xi | B\right) = \sum_{i = 1}^{n} x_i P\left(A_i | B\right) =
    \sum_{i = 1}^{n} x_i \frac{\mathbb{E} \left(\text{Ind}_{A_i} \text{Ind}_{B}\right)}{P\left(B\right)} = \frac{\mathbb{E}\left(\xi \text{Ind}_{B}\right)}{P\left(B\right)}
\]
Это выражение будем называть $\textit{условным математическим ожиданием относительно события B}$.

\end{definition}

Пусть теперь имеется разбиение 
\[
    \Omega = \bigcup_{k = 1}^{N} B_k, \ B_k \bigcap B_m = \emptyset, \ P\left(B_k\right) > 0
\]
Обозначим это разбиение $\{B_k\}$ через $\mathcal{B}$. Удобно собрать вместе значения условных математических ожиданий $\mathbb{E}\left(\xi | B_k\right)$.
\begin{definition}
Рассмотрим случайную величину: 
\[
    \Lambda\left(w\right) = \sum_{i = 1}^{N} \text{Ind}_{B_i}\left(w\right) \mathbb{E} \left(\xi | B_i\right)
\]
Если $w \in B_i$, то эта случайная величина выдает среднее значение $\xi$ при условии, что произошло событие $B_i$. Величину $\Lambda\left(w\right)$ называют $\textit{условным математическим ожиданием относительно разбиения}$ $ \mathcal{B}$ и обозначают через $\mathbb{E} \left(\xi | \mathcal{B}\right)$.
\end{definition}
Случайную величину
\[
    P\left(A | \mathcal{B}\right) = \mathbb{E}\left(\text{Ind}_A | \mathcal{B}\right)
\]
называют условной вероятностью события $A$ относительно разбиения $\mathcal{B}$. Ясно, что 
\[
    \mathbb{E}\left(\xi | \mathcal{B}\right) = \sum_{i = 1}^{N} \text{Ind}_{B_i}\left(w\right) \mathbb{E} \left(\xi | B_i\right) 
    = \sum_{i = 1}^{N} \text{Ind}_{B_i}\left(w\right) \sum_{j = 1}^{n} x_j P\left(A_j | B_i\right)
    = \sum_{j = 1}^{n} \sum_{i = 1}^{N} \text{Ind}_{B_i}\left(w\right) x_j P\left(A_j | B_i\right)
    = 
\]
\[
    = \sum_{j = 1}^{n} x_j \sum_{i = 1}^{N} \text{Ind}_{B_i}\left(w\right) P\left(A_j | B_i\right)
    = \sum_{j = 1}^{n} x_j \sum_{i = 1}^{n} \text{Ind}_{B_i}\left(w\right) \mathbb{E}\left(\text{Ind}_{A_j} | B_i\right) = \sum_{j = 1}^{n} x_j \mathbb{E} \left(\text{Ind}_{A_j} | \mathcal{B}\right)= 
\]
\[
    = \sum_{j = 1}^{n} x_j P\left(A_j | \mathcal{B}\right)
\]

\begin{example}
Рассмотрим важный пример, когда $\mathcal{B} = \left\{ B, \overline{B}\right\}$. Тогда 
\[
    P\left(A | \mathcal{B}\right) = \text{Ind}_{B} P\left(A | B\right) + \text{Ind}_{\overline{B}}P\left(A | \overline{B}\right)
\]
Если $w \in B$, то $P\left(A | \mathcal{B}\right)\left(w\right) = P\left(A | B\right)$
\end{example}

\begin{theorem}
Имеют место следующие свойства условного математического ожидания:
\begin{enumerate}[label=(\arabic*)]
\item (линейность) $\mathbb{E}\left(\alpha \xi + \beta \eta | \mathcal{B}\right) = \alpha \mathbb{E}\left(\xi | \mathcal{B}\right) + \beta \mathbb{E}\left(\eta | \mathcal{B}\right)$
\item (монотонность) из $\xi \leq \eta$ следует $\mathbb{E}\left(\xi | \mathcal{B}\right) \leq \mathbb{E}\left(\eta | \mathcal{B}\right)$
\item (аналог формулы полной вероятности) $\mathbb{E}\left(\mathbb{E}\left(\xi | \mathcal{B}\right)\right) = \mathbb{E}\xi$
\item (независимость) если случаная величина $\xi$ не зависит от разбиения $\mathcal{B}$, т.е. случайные величины $\xi$ и $\text{Ind}_{B_k}$ независимы, то $\mathbb{E}\left(\xi | \mathcal{B}\right) = \mathbb{E} \xi$
\item для всякой случайной величины $\eta = \sum_{k = 1}^{N} c_k \text{Ind}_{B_k}$ верно равенство
$\mathbb{E}\left(\eta \xi | \mathcal{B}\right) = \eta \mathbb{E}\left(\xi | \mathcal{B}\right)$

\end{enumerate}
\end{theorem}
\begin{proof}
Свойства $(1)$ и $(2)$ следуют из того, что они верны отдельно для каждого  $B_k$. \\ 
Свойство $(3)$ проверяется непосредственной подстановкой: 
\[
    \mathbb{E}\left(\mathbb{E}\left(\xi | \mathcal{B}\right)\right) = \mathbb{E} \left(\sum_{i = 1}^{N} \text{Ind}_{B_{i}} \mathbb{E}\left(\xi | B_i\right)\right)
    = \mathbb{E} \left(\sum_{i = 1}^{N} \text{Ind}_{B_{i}} \frac{\mathbb{E}\left(\xi \text{Ind}_{B_i}\right)}{P\left(B\right)}\right) = \sum_{i = 1}^{N} \mathbb{E}\left(\text{Ind}_{B_i}\right) \frac{\mathbb{E}\left(\xi \text{Ind}_{B_i}\right)}{P\left(B_i\right)} = \]
    \[
    =\sum_{i = 1}^{N} \mathbb{E}\left(\xi \text{Ind}_{B_i}\right) = \mathbb{E} \xi
\]
Обоснуем пункт $(4)$. Так как $\xi$ и $\text{Ind}_{B_k}$ независимы, то 
\[
    \mathbb{E}\left(\xi | B_k\right) = \frac{\mathbb{E}\left(\xi \text{Ind}_{B_k}\right)}{P\left(B_k\right)} = \frac{\mathbb{E}\xi \mathbb{E} \text{Ind}_{B_k}}{P\left(B_k\right)} = \mathbb{E}\xi
\]
Следовательно,
\[
    \mathbb{E}\left(\xi | \mathcal{B}\right) = \sum_{k = 1}^{N} \text{Ind}_{B_k}\left(w\right) \mathbb{E} \left(\xi | B_k\right) = \sum_{k = 1}^{N} \text{Ind}_{B_k}\left(w\right) \mathbb{E} \xi = \mathbb{E} \xi
\]
Для обоснования $(5)$ достаточно заметить, что
\[
    \mathbb{E}\left(\eta \xi | B_k\right) = c_k \mathbb{E}\left(\xi | B_k\right)
\]
\end{proof}

Наиболее типична ситуация, когда разбиение $\mathcal{B}$ появляется посредством некоторой случайной величины
\[
    \eta = \sum_{i = 1}^{N} y_i \text{Ind}_{B_i},
\]
где $y_i \ - $ разлиные числа и $P\left(B_i\right) > 0$.
\begin{definition}
В этом случае $B_i = \left\{w : \eta\left(w\right) = y_i\right\}$ и условное математическое ожидание $\mathbb{E}\left(\xi | \mathcal{B}\right)$ обозначают через $\mathbb{E}\left(\xi| \eta\right)$ и называют \textit{условным математическим ожиданием относительно} $\eta$.

\end{definition}
Несложно предъявить функцию $F$ (это можно сделать несколькими способами), что 
\[
    \mathbb{E}\left(\xi | \eta\right)\left(w\right) = F\left(\eta\left(w\right)\right)
\]
Легко видеть, что $F\left(y_i\right) = \mathbb{E}\left(\xi | B_i\right)$.
\\
Можно воспринимать $\mathbb{E}\left(\xi | \eta\right)$ как проекцию $\xi$ на $\eta$, а $\mathbb{E} \xi \eta$ как их скаляное произведение.
\begin{lemma}
Для условного математического ожидания выполнено
\[
    \mathbb{E}\left(\xi f\left(\eta\right)\right) = \mathbb{E}\left[f\left(\eta\right)\mathbb{E}\left(\xi | \eta\right)\right]
\]
для произвольной функции $f$. Кроме того, если для какой-то случайной величины $\zeta = g\left(\eta\right)$ выполнено 
\[
    \mathbb{E}\left(\xi f\left(\eta\right)\right) = \mathbb{E}\left(f\left(\eta\right) \zeta\right),
\]
то $\zeta = \mathbb{E}\left(\xi | \eta\right)$ п.н.
\end{lemma}

\begin{proof}
По $(5)$ и $(4)$ из теоремы 11:
\[
    \mathbb{E}\left[f\left(\eta\right) \mathbb{E}\left(\xi | \eta\right)\right] =
    \mathbb{E} \left[\mathbb{E}\left(f\left(\eta\right) \xi | \eta\right)\right] = \mathbb{E}\left(f\left(\eta\right) \xi\right)
\]
Докажем вторую часть: 
\[
    \mathbb{E} \left(f\left(\eta\right) \zeta\right) = \mathbb{E}\left(\xi f\left(\eta\right)\right) = \mathbb{E}\left[f\left(\eta\right)\mathbb{E}\left(\xi | \eta\right)\right]
\]
\[
    \mathbb{E} \left[f\left(\eta\right) \left(\zeta - \mathbb{E}\left(\xi | \eta \right)\right)\right] = 0
\]
Так как $\zeta$ и $\mathbb{E}\left(\xi | \eta\right) \ -$ функции от $\eta$, то возьмем $f\left(\eta\right) = \zeta - \mathbb{E}\left(\xi | \eta \right)$ и получим:
\[
    \mathbb{E} \left(\zeta - \mathbb{E}\left(\xi | \eta \right)\right)^ 2 = 0,
\]
то есть $\zeta = \mathbb{E}\left(\xi | \eta\right)$ п.н.
\end{proof}
Теперь докажем, что $\mathbb{E}\left(\xi | \eta\right)$ и правда является проекцией $\xi$ на $\eta$.
\begin{advice}

Пусть $\mathbb{E} \xi ^ 2 < \infty$. Условное матетическое ожидание $\mathbb{E}\left(\xi | \eta\right)$ среди всех случайных величин вида $f\left(\eta\right)$ является лучшим среднеквадратическим приближением для $\xi$, т.е.
\[
    \min_{\zeta : \zeta = f\left(\eta\right)} \mathbb{E} \left(\xi - \zeta\right) ^ 2 = \mathbb{E} \left[\xi - \mathbb{E}\left(\xi | \eta\right)\right] ^ 2
\]
\end{advice}

\begin{proof}
Пусть $\zeta = f\left(\eta\right)$. Так как ($\mathbb{E} \left(\xi | \eta\right) - \zeta \ -$ функция от $\eta$) 
\[
    \mathbb{E} \left[\left(\xi - \mathbb{E}\left(\xi | \eta\right)\right)\left(\mathbb{E} \left(\xi | \eta\right) - \zeta\right)\right] = 0,
\]
то
\[
    \mathbb{E} \left(\xi - \zeta\right) ^2 = \mathbb{E} \left[\left(\xi - \mathbb{E}\left(\xi | \eta\right)\right) + \left(\mathbb{E}\left(\xi | \eta\right) - \zeta\right)\right] ^ 2
    = \]
    \[
    \mathbb{E}\left[\xi - \mathbb{E}\left(\xi | \eta\right)\right] ^ 2 + 2 \mathbb{E} \left[\left(\xi - \mathbb{E}\left(\xi | \eta\right)\right)\left(\mathbb{E} \left(\xi | \eta\right) - \zeta\right)\right] + \mathbb{E} \left[\mathbb{E}\left(\xi | \eta\right) - \zeta\right] ^ 2 \geq \mathbb{E} \left[\xi - \mathbb{E}\left[\xi | \eta\right)\right] ^ 2
\]
Последнее неравенство достигается взятием $\zeta = \mathbb{E}\left(\xi | \eta\right)$
\end{proof}

\clearpage

\section{Условные математические ожидания: общий случай}

\begin{definition}
$\xi, \eta \ -$ случаные величины. $\mathbb{E} \abs{\xi} < \infty$. Тогда случайная величина вида $F\left(\eta\right)$ называется $\textit{условным математическим ожиданием} \mathbb{E}\left(\xi | \eta\right)$, если 
\[
    \mathbb{E} \left[\xi f\left(\eta\right)\right] = \mathbb{E} \left[\mathbb{E} \left(\xi | \eta\right) f\left(\eta\right)\right]
\]
для любой ограниченной $f$. Любые две случаные величины, удовлетворяющие этому условию почти наверное совпадают (лемма 3).
\end{definition}
Из этого определения следует, что $\mathbb{E}\left(\xi | \eta\right)$  есть наименее отличающаяся от $\xi$ случайная величина вида $F\left(\eta\right)$, то есть проекция $\xi$ на $\eta$.

\begin{advice}
Предположим, что распределение случайной величины $\left(\xi, \eta\right)$ задано совместной плотностью $\rho_{\xi \eta}\left(x, y\right)$. Тогда
\[
    \mathbb{E} \left(g\left(\xi, \eta\right) | \eta = y \right) = \int_{-\infty}^{+\infty} g\left(x, y\right)\frac{\rho_{\xi \eta}\left(x, y\right)}{\rho_{\eta}\left(y\right)} dx
\]
\end{advice} 

\begin{proof}
Имеет место цепочка равенств:
\[
    \mathbb{E} \left[g\left(\xi, \eta\right) f\left(\eta\right)\right] = \int_{-\infty}^{+\infty} \int_{-\infty}^{+\infty} g\left(x, y\right) f\left(y\right) \rho_{\xi \eta}\left(x, y\right) dx dy  = \]
    \[
    = \int_{-\infty}^{+\infty} f\left(y\right) \underbrace{\int_{-\infty}^{+\infty}g\left(x, y\right)\frac{\rho_{\xi \eta}\left(x, y\right)}{\rho_{\eta}\left(y\right)}}_{= F\left(y\right)} \rho_{\eta}\left(y\right) dy = \mathbb{E} \left[F\left(\eta\right) f\left(\eta\right)\right]
\]
Следовательно $F\left(\eta\right) = \mathbb{E} \left(\xi | \eta\right)$ по определению.
\end{proof}
\end{document}
