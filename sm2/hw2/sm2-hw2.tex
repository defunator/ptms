\usepackage{amsmath}
\usepackage{amsfonts}
\usepackage{amssymb}
\usepackage{graphicx}
\usepackage{blindtext}
\usepackage{textcomp}
\usepackage{pgfplots}

\pgfplotsset{width=10cm,compat=1.9}


%Header styles
\usepackage{fancyhdr}
\setlength{\headheight}{15pt}
\pagestyle{fancy}
\renewcommand{\sectionmark}[1]{\markright{#1}{}}
\fancyhf{}
\fancypagestyle{plain}{ %
\fancyhf{} % remove everything
\renewcommand{\headrulewidth}{0pt} % remove lines as well
\renewcommand{\footrulewidth}{0pt}}

%makes available the commands \proof, \qedsymbol and \theoremstyle
\usepackage{amsthm}

%Ruler
\newcommand{\HRule}{\rule{\linewidth}{0.5mm}}

%Commands for naturals, integers, topology, hull, Ball, Disc, Dimension, boundary and a few more
\newcommand{\E}{{\mathcal{E}}}
\newcommand{\F}{{\mathcal{F}}}
\newcommand{\T}{{\mathcal{T}}}
\newcommand{\Bs}{{\mathcal{B}}}
\newcommand{\R}{{\mathbb{R}}}
\newcommand{\Q}{{\mathbb{Q}}}
\newcommand{\Z}{{\mathbb{Z}}}
\newcommand{\Nt}{{\mathbb{N}}}
\newcommand{\B}{{\mathbf{B}}}
\renewcommand{\S}{{\mathbf{S}}}
\newcommand{\K}{{\mathfrak{C}}}
\newcommand{\N}{{\mathfrak{N}}}
\newcommand{\I}{{\mathbf{I}}}
\newcommand{\dime}{{\rm dim}\,}
\newcommand{\est}{{\rm Est}\,}
\newcommand{\inte}{{\rm int}\,}
\newcommand{\conv}{{\rm conv}\,}
\renewcommand{\max}{{\rm sup\,}}
\newcommand{\diam}{{\rm di\acute{a}m\,}}
\newcommand{\leyenda}[1]{\caption{{\small \textsf{#1}}}}
\renewcommand{\inf}{{\rm \acute{i}nf}\,}

\usepackage[margin=1in]{geometry} 
\usepackage{amsmath}
\usepackage{tcolorbox}
\usepackage{indentfirst}
\usepackage{amssymb}
\usepackage{amsthm}
\usepackage{lastpage}
\usepackage{fancyhdr}
\usepackage{enumitem}
\usepackage{accents}
\usepackage{blindtext}
\usepackage{booktabs}
\usepackage{enumitem}
\usepackage{mathtools}
\pagestyle{fancy}
\setlength{\headheight}{40pt}
\usepackage[utf8]{inputenc}
\usepackage[russian]{babel}
\everymath{\displaystyle}
\usepackage{cancel}
\geometry{verbose,a4paper,tmargin=2cm,bmargin=2cm,lmargin=1cm,rmargin=1.5cm}

\usepackage{hyperref}
\hypersetup{
    colorlinks,
    citecolor=black,
    filecolor=black,
    linkcolor=blue,
    urlcolor=blue
}

\newenvironment{solution}
  {\renewcommand\qedsymbol{$\blacksquare$}
  \begin{proof}[Solution]}
  {\end{proof}}
\renewcommand\qedsymbol{$\blacksquare$}

\newtheorem{theorem}{Теорема}
\newtheorem{lemma}{Лемма}
\newtheorem{statement}{Утверждение}
\newtheorem{corollary}{Следствие}
\newtheorem{advice}{Предложение}

\theoremstyle{definition}
\newtheorem{example}{Пример}

\theoremstyle{definition}
\newtheorem{definition}{Определение}

\theoremstyle{remark}
\newtheorem{remark}{Замечание}

\theoremstyle{definition}
\newtheorem*{reminder}{Напоминание}

\newcommand{\ubar}[1]{\underaccent{\bar}{#1}}

\DeclarePairedDelimiter\abs{\lvert}{\rvert}%

\makeatletter
\let\oldabs\abs
\def\abs{\@ifstar{\oldabs}{\oldabs*}}



\begin{document}
\lhead{Иванов Семен} 
\rhead{БПМИ-183} 
\cfoot{\thepage\ of \pageref{LastPage}}

\section{Листок 1. Задача 12}

\begin{itemize}
\item Найдем функцию распределения $M_n$. Утверждение $M_n \leq x$ равносильно тому, что $\xi_1, \cdots \xi_n \leq t$.
Таким образом, получаем: 
\[
    F_{M_n}\left(x\right) = P\left(M_n \leq x\right) = P\left(\xi_1 \leq x\right) \cdots P\left(\xi_n \leq x\right) = F^n\left(x\right)
\]
\item $F\left(x\right) \sim 1 - \frac{b}{x ^ a}$ при $x \to \infty$, так как $\lim_{x \to \infty}x ^ a \left(1 - F\left(x\right)\right) = b$, где $a, b > 0$.
\item Пусть $\eta = \frac{M_n}{\left(bn\right) ^ \frac1a}$. Тогда имеем: 
\[
    \lim_{n \to \infty} F_{\eta}\left(x\right) = \lim_{n \to \infty} P\left(M_n \leq x \cdot \left(bn\right) ^ \frac1a\right) = \lim_{n \to \infty} F ^ n\left(x \cdot \left(bn\right) ^ \frac1a\right)
\]
Теперь воспользуемся полученной эквивалентностью (при $x \geq 0$, при $x < 0$ по определению функции распределения ясно, что $F_{\eta}\left(x\right) \xrightarrow{n \to \infty} 0$, так как $\lim_{x \to -\infty} F_{\eta}\left(x\right) = 0$):
\[
    \lim_{n \to \infty} F ^ n\left(x \cdot \left(bn\right) ^ \frac1a\right) = \lim_{n \to \infty} \left(1 - \frac{b}{x ^ a \cdot \left(bn\right)}\right) ^ n = \lim_{n \to \infty} \left(1 - \frac{1}{x ^ a \cdot n}\right) ^ {\left(-x ^ a \cdot n \right) \cdot \left(-x ^ {-a}\right)} = e ^ {-x ^ {-a}}
\]
\qed
\end{itemize}

\section{Листок 2. Задача 5a}
\begin{itemize}
\item Потому что не выполняется свойство характеристических функций, что $\phi\left(0\right) = 1$.
\end{itemize}

\section{Листок 2. Задача 10a}
\begin{itemize}
\item Посчитаем матожидание по определению:
\[
    \phi_{\eta}\left(x\right) = E e ^ {ix\eta} = \int_{0}^{1} e ^ {ixt} \left(1 - t\right) dt + \int_{-1}^{0} e ^ {ixt} \left(1 + t\right) dt
\]
\item $\int e ^ {ixt} dt = \frac{e ^ {ixt}}{ix} + C$
\item $\int e ^ {ixt} t dt = \frac{1}{ix} \int t d\left(e ^ {ixt}\right) = \frac{1}{ix} \cdot t e ^ {ixt} - \frac{1}{ix} \int e ^ {ixt} dt = \frac{e ^ {ixt}}{ix} \left(t - \frac{1}{ix}\right) + C$
\item Продолжая первый пункт: 
\[
    \phi_{\eta}\left(x\right) = \frac{1}{ix} \left(e ^ {ix} - 1\right) - \frac{e ^ {ix} \left(ix - 1\right) + 1}{\left(ix\right) ^ 2} + \frac{1}{ix} \left(1 - e ^ {-ix}\right) + \frac{-1 - e ^ {-ix} \left(-ix - 1\right)}{\left(ix\right) ^ 2}
    = \frac{e ^ {ix} - 2 + e ^ {-ix}}{\left(ix\right) ^ 2}=
\]
\[
   =\frac{-e ^ {-ix} + 2 - e ^ {ix}}{x ^ 2} 
\]
\end{itemize}

\section{Листок 2. Задача 3}
\begin{itemize}
\item $\varphi_{a\xi + b}\left(t\right) = E e ^ {it \left(a \xi + b\right)} = e ^ {itb} E e ^ {i\left(at\right)\xi} = e ^ {itb} \varphi_{\xi}\left(at\right)$
\end{itemize}
\begin{enumerate}[label=(\alph*)]
\item Соберем случайную величину с распределением $U[a, b]$ из линейной комбинации $\xi$ с распределением $U[0, 1]$. Это будет просто $\left(b-a\right) \xi + a$ (биекция из отрезка $[0,1]$ в $[a, b]$, несложно видеть, что $F_{\left(b - a\right) \xi + a}\left(t\right) = P\left(\xi \leq \frac{t - a}{b - a}\right) = \frac{t - a}{b - a}$, то есть случайная величина имеет нужное распределение). Получаем:
\[
\varphi_{\left(b - a\right) \xi + a}\left(t\right) e ^ {ita} \varphi_{\xi}\left(\left(b - a\right) t\right) = e ^ {ita} \frac{e ^ {it\left(b - a\right)} - 1}{it\left(b - a\right)} = \frac{e ^ {itb} - e ^ {ita}}{it\left(b - a\right)}
\]
\item Соберем случайную величину с распределением $N(\mu, \sigma ^ 2)$ из линейной комбинации $\xi$ с распределением $N(0, 1)$. Это будет просто $\sigma \xi + \mu$. Матожидание и дисперсия вроде нужные, проверим функцию распределения: 
$$F_{\sigma \xi + \mu}\left(t\right) = P\left(\xi \leq \frac{t - \mu}{\sigma}\right) = \frac{1}{\sqrt{2 \pi}} \int_{- \infty}^{\frac{t - \mu}{\sigma}} e ^ {\frac{x ^ 2}{2}} dx = $$
сделаем замену $y = \sigma x + \mu$, получим: $$= \frac{1}{\sigma \sqrt{2 \pi} } \int_{-\infty}^{t} e ^ {\left(\frac{y - \mu}{\sigma}\right) ^ 2 \cdot \frac{1}{2}} dy$$
Теперь уже видно, что если взять производную, то получим нужную плотность $\frac{1}{\sigma \sqrt{2 \pi}} \cdot e ^ {\frac{\left(t - y\right) ^ 2}{\sigma ^ 2}}$ \\
Теперь вычислим характеристическую функцию: 
\[
    \varphi_{\sigma \xi + \mu}\left(t\right) = e ^ {it \mu} \varphi_{\xi}\left(\sigma t\right) = e ^ {it \mu} \cdot e ^ {-\frac{\sigma ^ 2 t ^ 2}{2}} = e ^ {it\mu - \frac{\sigma ^ 2 t ^ 2}{2}}
\]
\end{enumerate}

\section{Листок 2. Задача 11}
\begin{itemize}
    \item Возьмем плотность из $10a$
    \item Найдем хар функцию: 
    \[
        \varphi\left(t\right) = E e ^ {it\xi} = E \cos\left(t \xi\right) + i E \sin\left(t \xi\right) = E \cos\left(t \xi\right) + i \int_{-1}^{1} \left(1 - \abs{x}\right) \sin\left(t x\right) dx
    \]
    Так как этот интеграл от нечетной функции по симметричному промежутку, то он $= 0$. Следовательно имеем:
    \[
        \varphi\left(t\right) = E \cos\left(t \xi\right)
    \]
\end{itemize}



\end{document}
