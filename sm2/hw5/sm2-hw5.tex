\usepackage{amsmath}
\usepackage{amsfonts}
\usepackage{amssymb}
\usepackage{graphicx}
\usepackage{blindtext}
\usepackage{textcomp}
\usepackage{pgfplots}

\pgfplotsset{width=10cm,compat=1.9}


%Header styles
\usepackage{fancyhdr}
\setlength{\headheight}{15pt}
\pagestyle{fancy}
\renewcommand{\sectionmark}[1]{\markright{#1}{}}
\fancyhf{}
\fancypagestyle{plain}{ %
\fancyhf{} % remove everything
\renewcommand{\headrulewidth}{0pt} % remove lines as well
\renewcommand{\footrulewidth}{0pt}}

%makes available the commands \proof, \qedsymbol and \theoremstyle
\usepackage{amsthm}

%Ruler
\newcommand{\HRule}{\rule{\linewidth}{0.5mm}}

%Commands for naturals, integers, topology, hull, Ball, Disc, Dimension, boundary and a few more
\newcommand{\E}{{\mathcal{E}}}
\newcommand{\F}{{\mathcal{F}}}
\newcommand{\T}{{\mathcal{T}}}
\newcommand{\Bs}{{\mathcal{B}}}
\newcommand{\R}{{\mathbb{R}}}
\newcommand{\Q}{{\mathbb{Q}}}
\newcommand{\Z}{{\mathbb{Z}}}
\newcommand{\Nt}{{\mathbb{N}}}
\newcommand{\B}{{\mathbf{B}}}
\renewcommand{\S}{{\mathbf{S}}}
\newcommand{\K}{{\mathfrak{C}}}
\newcommand{\N}{{\mathfrak{N}}}
\newcommand{\I}{{\mathbf{I}}}
\newcommand{\dime}{{\rm dim}\,}
\newcommand{\est}{{\rm Est}\,}
\newcommand{\inte}{{\rm int}\,}
\newcommand{\conv}{{\rm conv}\,}
\renewcommand{\max}{{\rm sup\,}}
\newcommand{\diam}{{\rm di\acute{a}m\,}}
\newcommand{\leyenda}[1]{\caption{{\small \textsf{#1}}}}
\renewcommand{\inf}{{\rm \acute{i}nf}\,}

\usepackage[margin=1in]{geometry} 
\usepackage{amsmath}
\usepackage{tcolorbox}
\usepackage{indentfirst}
\usepackage{amssymb}
\usepackage{amsthm}
\usepackage{lastpage}
\usepackage{fancyhdr}
\usepackage{enumitem}
\usepackage{accents}
\usepackage{blindtext}
\usepackage{booktabs}
\usepackage{enumitem}
\usepackage{mathtools}
\pagestyle{fancy}
\setlength{\headheight}{40pt}
\usepackage[utf8]{inputenc}
\usepackage[russian]{babel}
\everymath{\displaystyle}
\usepackage{cancel}
\geometry{verbose,a4paper,tmargin=2cm,bmargin=2cm,lmargin=1cm,rmargin=1.5cm}

\usepackage{hyperref}
\hypersetup{
    colorlinks,
    citecolor=black,
    filecolor=black,
    linkcolor=blue,
    urlcolor=blue
}

\newenvironment{solution}
  {\renewcommand\qedsymbol{$\blacksquare$}
  \begin{proof}[Solution]}
  {\end{proof}}
\renewcommand\qedsymbol{$\blacksquare$}

\newtheorem{theorem}{Теорема}
\newtheorem{lemma}{Лемма}
\newtheorem{statement}{Утверждение}
\newtheorem{corollary}{Следствие}
\newtheorem{advice}{Предложение}

\theoremstyle{definition}
\newtheorem{example}{Пример}

\theoremstyle{definition}
\newtheorem{definition}{Определение}

\theoremstyle{remark}
\newtheorem{remark}{Замечание}

\theoremstyle{definition}
\newtheorem*{reminder}{Напоминание}

\newcommand{\ubar}[1]{\underaccent{\bar}{#1}}

\DeclarePairedDelimiter\abs{\lvert}{\rvert}%

\makeatletter
\let\oldabs\abs
\def\abs{\@ifstar{\oldabs}{\oldabs*}}



\begin{document}

\lhead{Иванов Семен} 
\rhead{БПМИ-183} 
\cfoot{\thepage\ of \pageref{LastPage}}

\section{Листок 3. Задача 8c1}
\begin{itemize}
\item Применим метод Грама-Шмидта к подпространству $\left(X_1, X_2\right)$ со скалярным произведением $\left(X, Y\right) = E XY$:
\[
    b_1 = X_1
\]
\[
    b_2 = X_2 - pr_{b_1} X_2 = X_2 - \frac{\left(X_2, X_1\right)}{\left(X_1, X_1\right)}X_1 = X_2 - \frac{\text{cov}\left(X_2, X_1\right)}{\text{cov}\left(X_1, X_1\right)} X_1 = -\frac 1 2 X_1 + X_2
\]
Теперь отнормируем вектора:
\[
    Y_1 = \frac{b_1}{\sqrt{\left(b_1, b_1\right)} }= \frac{b_1}{2} = \frac{X_1}{\sqrt{2}}
\]
\[
    Y_2 = \frac{b_2}{\sqrt{\left(b_2, b_2\right)}} = \frac{b_2}{\sqrt{\frac 1 4 \cdot 2 + 1 - 2 \cdot \frac 1 2 \cdot 1}} = \sqrt{2}b_2 = -\frac{X_1}{\sqrt{2}} + \sqrt{2} X_2
\]

\item Теперь имеем:
\[
    \begin{pmatrix}
    \frac {1} {\sqrt {2}} & 0 \\
    -\frac {1} {\sqrt{2}} & \sqrt {2}
    \end{pmatrix}
    \begin{pmatrix}
    X_1 \\
    X_2
    \end{pmatrix} = 
    \begin{pmatrix}
    Y_1 \\
    Y_2
    \end{pmatrix}
\]
\[
    \begin{pmatrix}
    X_1 \\
    X_2
    \end{pmatrix} = 
    \begin{pmatrix}
    \sqrt 2 & 0 \\
    1 & \frac{ 1}{ \sqrt{2}}
    \end{pmatrix}
    \begin{pmatrix}
    Y_1 \\
    Y_2
    \end{pmatrix}
\]

\item $Y_1, Y_2 \sim N\left(0, 1\right)$ по теореме с лекции.
\end{itemize}

\section{Листок 3. Задача 11a}
\begin{itemize}
\item Так как величины независимы:
\[
    \rho_{\xi, \eta}\left(x, y\right) = \rho_{\xi}\left(x\right) \rho_{\eta}\left(y\right) = \frac 1 {8 \pi} e ^ {-\frac{x ^ 2 + y ^ 2}{8}}
\]
\item Распишем искомую веротяность, как интеграл плотности по нужной области $G$ (разности окружности радиуса 3 и 2) и сделаем тригонометрическую замену ($x = r \cos \varphi$, $y = r \sin \varphi$):
\[
    P\left(4 \leq \xi ^ 2 + \eta ^ 2 \leq 9\right) = \int_{G} \rho_{\xi, \eta}\left(x, y\right) dxdy = \int_{0}^{2 \pi} \int_{2}^{3} \frac 1{8 \pi} e ^ {-\frac{r ^ 2}{8}} dr d \varphi = 2 \pi \cdot \frac 1 {8\pi} \int_{2}^{3} r e ^ {-\frac{r ^ 2}{8}}dr = 
\]
\[
    =  \frac{1}{8} \int_{2}^{3} e ^ {-\frac{r ^ 2}{8}} d\left(r ^ 2\right) = \frac1 8 \cdot \left(-8\right) \left(e ^ {-\frac 9 8} - e ^ {- \frac 1 2}\right) =  \left(e ^ {- \frac 1 2} - e ^ {-\frac 9 8}\right)
\]

\end{itemize}

\section{Задача 13}

\begin{itemize}
\item Выпишем интеграл матожидания и посчитаем его:
\[
    \mathbb{E} \xi ^ 4 = \frac {1}{\sqrt{2 \pi}} \int_{-\infty}^{+\infty} x ^ 4 e ^ {-\frac{x ^ 2}{2}} dx = 
    -\frac{1}{\sqrt{2 \pi}} \int_{-\infty}^{+\infty} x ^ 3 d \left(e ^ {-\frac{x ^ 2}{2}}\right) = 
    -\frac{1}{\sqrt{2 \pi}} \left(x ^ 3 e ^ {-\frac{x ^ 2}{2}}\right)\Big|_{-\infty}^{+\infty} + \frac{3}{\sqrt{2\pi}} \int_{-\infty}^{+\infty} x ^ 2 e ^ {-\frac{x ^ 2}{2}} dx = 
\]
\[
    = 0 - \frac{3}{\sqrt{2 \pi}} \int_{-\infty}^{+\infty} x d\left(e ^ {-\frac{x ^ 2}{2}}\right) = 
    -\frac{3}{\sqrt{2 \pi}} \left(x e ^ {-\frac{x ^ 2}{2}}\right)\Big|_{-\infty}^{+\infty} + \frac{3}{\sqrt{2 \pi}} \int_{-\infty}^{+\infty} e ^ {-\frac{x ^ 2}{2}} dx =
    0 + 3 \cdot \underbrace{\frac{1}{\sqrt{2 \pi}} \int_{-\infty}^{+\infty} e ^ {-\frac{x ^ 2}{2}}dx}_{= \Phi\left(+\infty\right) - \Phi\left(- \infty\right) = 1 - 0} = 3 
\]
\end{itemize}

\end{document}