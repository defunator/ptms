\usepackage{amsmath}
\usepackage{amsfonts}
\usepackage{amssymb}
\usepackage{graphicx}
\usepackage{blindtext}
\usepackage{textcomp}
\usepackage{pgfplots}

\pgfplotsset{width=10cm,compat=1.9}


%Header styles
\usepackage{fancyhdr}
\setlength{\headheight}{15pt}
\pagestyle{fancy}
\renewcommand{\sectionmark}[1]{\markright{#1}{}}
\fancyhf{}
\fancypagestyle{plain}{ %
\fancyhf{} % remove everything
\renewcommand{\headrulewidth}{0pt} % remove lines as well
\renewcommand{\footrulewidth}{0pt}}

%makes available the commands \proof, \qedsymbol and \theoremstyle
\usepackage{amsthm}

%Ruler
\newcommand{\HRule}{\rule{\linewidth}{0.5mm}}

%Commands for naturals, integers, topology, hull, Ball, Disc, Dimension, boundary and a few more
\newcommand{\E}{{\mathcal{E}}}
\newcommand{\F}{{\mathcal{F}}}
\newcommand{\T}{{\mathcal{T}}}
\newcommand{\Bs}{{\mathcal{B}}}
\newcommand{\R}{{\mathbb{R}}}
\newcommand{\Q}{{\mathbb{Q}}}
\newcommand{\Z}{{\mathbb{Z}}}
\newcommand{\Nt}{{\mathbb{N}}}
\newcommand{\B}{{\mathbf{B}}}
\renewcommand{\S}{{\mathbf{S}}}
\newcommand{\K}{{\mathfrak{C}}}
\newcommand{\N}{{\mathfrak{N}}}
\newcommand{\I}{{\mathbf{I}}}
\newcommand{\dime}{{\rm dim}\,}
\newcommand{\est}{{\rm Est}\,}
\newcommand{\inte}{{\rm int}\,}
\newcommand{\conv}{{\rm conv}\,}
\renewcommand{\max}{{\rm sup\,}}
\newcommand{\diam}{{\rm di\acute{a}m\,}}
\newcommand{\leyenda}[1]{\caption{{\small \textsf{#1}}}}
\renewcommand{\inf}{{\rm \acute{i}nf}\,}

\usepackage[margin=1in]{geometry} 
\usepackage{amsmath}
\usepackage{tcolorbox}
\usepackage{indentfirst}
\usepackage{amssymb}
\usepackage{amsthm}
\usepackage{lastpage}
\usepackage{fancyhdr}
\usepackage{enumitem}
\usepackage{accents}
\usepackage{blindtext}
\usepackage{booktabs}
\usepackage{enumitem}
\usepackage{mathtools}
\pagestyle{fancy}
\setlength{\headheight}{40pt}
\usepackage[utf8]{inputenc}
\usepackage[russian]{babel}
\everymath{\displaystyle}
\usepackage{cancel}
\geometry{verbose,a4paper,tmargin=2cm,bmargin=2cm,lmargin=1cm,rmargin=1.5cm}

\usepackage{hyperref}
\hypersetup{
    colorlinks,
    citecolor=black,
    filecolor=black,
    linkcolor=blue,
    urlcolor=blue
}

\newenvironment{solution}
  {\renewcommand\qedsymbol{$\blacksquare$}
  \begin{proof}[Solution]}
  {\end{proof}}
\renewcommand\qedsymbol{$\blacksquare$}

\newtheorem{theorem}{Теорема}
\newtheorem{lemma}{Лемма}
\newtheorem{statement}{Утверждение}
\newtheorem{corollary}{Следствие}
\newtheorem{advice}{Предложение}

\theoremstyle{definition}
\newtheorem{example}{Пример}

\theoremstyle{definition}
\newtheorem{definition}{Определение}

\theoremstyle{remark}
\newtheorem{remark}{Замечание}

\theoremstyle{definition}
\newtheorem*{reminder}{Напоминание}

\newcommand{\ubar}[1]{\underaccent{\bar}{#1}}

\DeclarePairedDelimiter\abs{\lvert}{\rvert}%

\makeatletter
\let\oldabs\abs
\def\abs{\@ifstar{\oldabs}{\oldabs*}}



\begin{document}

\lhead{Иванов Семен} 
\rhead{БПМИ-183} 
\cfoot{\thepage\ of \pageref{LastPage}}

\section{Листок 2. Задача 10b}
\begin{itemize}
\item $\varphi_{\eta}\left(t\right) = E e ^ {it\eta} = E e ^ {it \left(AU + \left(1 - A\right)V\right)}$
\item Распишем матожидание по индикаторам для случайной величины $A$:
\[
    \varphi_{\eta}\left(t\right) = P\left(A = 1\right) E e ^ {itU} + P\left(A = 0\right) E e ^ {itV} = p \varphi_{u}\left(t\right) + \left(1 - p\right) \varphi_{v}\left(t\right)
\]
\end{itemize}

\section{Листок 2. Задача 12}
\begin{theorem}(Слабый закон больших чисел) \\
Пусть $\xi_n \ -$ последовательность независимых одинаково распределенных случайных величин, причем $E \xi_1 = \mu < \infty, \ E \xi_1 ^ 2 < \infty$. Тогда для $\eta_n = \frac{\xi_1 + \cdots + \xi_n}{n}$:
\[
    \lim_{n \to \infty} P\left(\abs{\eta_n - \mu}\right) = 0
\]

\end{theorem}
\begin{proof}
Разложим хар функцию в ряд Маклорена до второго члена (можем это сделать, так как первые и вторые моменты конечны и воспользуемся формулой $\varphi_{X}^{\left(k\right)}\left(0\right) = i ^ k E X ^ k$):
$$\varphi_{X}\left(t\right) = 1 + it E X  + o\left(t\right)$$
Запишем разложение для $\eta_n$:
$$\varphi_{\eta_n}\left(t\right) = \varphi_{\xi_1 + \cdots + \xi_n}\left(\frac{t}{n}\right) = \varphi^{n}_{\xi_1}\left(\frac{t}{n}\right) = \left(1 + i \frac{t}{n} \mu + o \left(\frac{t}{n}\right)\right) ^ n = $$
$$ = \left(1 + \frac{it\mu}{n} + o\left(\frac{t}{n}\right)\right) ^ {\left(\frac{n}{it \mu}\right) \cdot \left(it \mu\right)} \xrightarrow{n \to \infty} e ^ {it\mu}$$
Получили характеристическую функцию константы, а значит функция распределенения будет совпадать с константной, то есть $\eta_n \xrightarrow{\text{d}} \mu$ $\Rightarrow$ $\eta_n \xrightarrow{\text{P}} \mu$. Из определения сходимости по вероятности, получаем:
\[
    \lim_{n \to \infty} P\left(\abs{\eta_n - \mu}\right) = 0
\]
\end{proof}

\end{document}
